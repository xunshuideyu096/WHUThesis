% !Mode:: "TeX:UTF-8"

%%% 说明: 此部分需要自己填写的内容:  (1) 中文摘要及关键词 (2) 英文摘要及关键词

%%%%%%%%%%%%%%%%%%%%%%%
%%% ------------ 中文摘要 ---------------%%%
%%%%%%%%%%%%%%%%%%%%%%%
\begin{cnabstract}

紫外光电探测器因具有在民用与军事领域上的巨大应用前景而备受研究人员的关注。宽带隙无机物——尤其是可使用低成本简单工艺制备的氧化锌(ZnO)——是制作紫外光电探测器的理想半导体材料。但由于p型ZnO材料不稳定,基于ZnO同质结的紫外光电探测器较少被报道。而将ZnO和有机半导体复合使用,为解决p型ZnO材料不稳定的问题提供了一种替代性方案。但是有机半导体较低的载流子迁移率,一定程度上限制了器件性能的提高。

金属卤化物钙钛矿材料具有带隙可调、载流子迁移率较高等优异的光电性能,被广泛应用于制作高性能的光电器件。其中氯化物钙钛矿具有合适的带隙宽度,可用于紫外光电探测器。但是有关基于稳定性更好的全无机钙钛矿的器件的报道相对较少,这是由于全无机钙钛矿前驱体在常规有机溶剂中溶解度较低,难以通过成熟的溶液法来制备,阻碍了它在紫外光电探测器上的应用。

针对上述问题,本文以提高基于ZnO与有机半导体复合层的紫外光电探测器的器件性能为出发点,研究宽带隙全无机钙钛矿材料的制备工艺及其在紫外光电探测器方面的应用,进行了以下探索:

其一,通过过饱和溶液再结晶方法制备\ce{CH3NH3PbCl3}钙钛矿量子点,并将其掺杂到聚[双(4-苯基)(2,4,6-三甲基苯基)胺](PTAA)聚合物导电薄膜中。研究表明,向PTAA中加入适量钙钛矿量子点,可在不影响器件暗电流密度的情况下提高光电流密度,并提升器件的响应度和比探测度。此外,\ce{CH3NH3PbCl3}量子点的引入可提高PTAA薄膜的空穴载流子迁移率,更有利于电荷的传输和抽取,进而提升紫外探测器的光电性能。

其二,尝试将全无机钙钛矿用作紫外光电探测器的光吸收层,使用热蒸发方式制备\ce{CsPbCl3}薄膜。研究表明,在随着对\ce{CsPbCl3}薄膜退火温度的增加,薄膜的晶粒增大、结晶性变好,并且紫外光电探测器的外量子效率从不足5\%上升至20\%左右。而使用较高温度对\ce{CsPbCl3}钙钛矿薄膜退火处理后,ZnO薄膜中的醋酸根离子减少,使可抑制钙钛矿薄膜中缺陷的相互作用减弱,导致紫外光电探测器性能下降。这一研究为制作基于宽带隙全无机钙钛矿(如\ce{CsPbCl3})薄膜的光电器件奠定了一定基础。

\end{cnabstract}
\vspace{1em}\par\vfill

%%%--------- 关键词 -------- %%%
\cnkeywords{{\heiti\zihao{-4}紫外光电探测器,宽带隙钙钛矿材料,钙钛矿量子点, 无机钙钛矿} }

%%%%%%%%%%%%%%%%%%%%%%%


%%%%%%%%%%%%%%%%%%%%%%%
%%% ------------ 英文摘要 ---------------%%%
%%%%%%%%%%%%%%%%%%%%%%%

\begin{enabstract}%This thesis is a study on the applications of perovskite materials on ultraviolet photodetectors\dots.

Ultraviolet (UV) photodetectors have attracted much attention for their wide applications in civilian and military fields. Wide bandgap inorganic semiconductor{\textemdash}zinc oxide (ZnO) could be fabricated with low-cost methods, and have emerged as a promising candidate for UV detection and imaging. Nevertheless, ZnO based homojuction photodetectors have been rarely reported owing to the instability of p-type ZnO. Combing organic semiconductor materials with ZnO can effectively solve this issue. However, this type of photodetectors usually suffers from the relatively low responsivity, due to the inferior charge transport properties.

Metal halide perovskite have been widely used in many high-performance optoelectronic devices due to their brilliant properties, such as bandgap adjustability and large carrier mobility. Furthermore, chlorine based organic-inorganic hybrid perovskites with suitable band gap have been introduced for UV photodetection. However, efficient UV photodetectors based on inorganic perovskites have been barely reported due to their poor solubility.

To solve those problems, we incorporated the \ce{CH3NH3PbCl3} quantum dots (QDs) with high carrier mobility into thick organic semiconductor layer to enhance the film charge transport properties and device performance. Besides, we also investigated the possibility of replacing organic layer directly by inorganic perovskite as the active layer in UV photodetectors. The key elements of this thesis is as follows:

1.\ \ \ce{CH3NH3PbCl3} perovskite QDs prepared via supersaturated recrystallization were doped into Poly(triarylamine) (PTAA) organic semiconductor films. With proper doping amount of \ce{CH3NH3PbCl3} QDs into PTAA, the responsivity of the UV devices could be increased without distinct change of the dark current density, and the specific detectivity of photodetectors could be improved as well. Furthemore, the hole mobility of doped PTAA was determined to be enhanced, that may facilitate the charge transport, and result improved device performance. In addition, the great long-time stability and flexibility of these devices also be demonstrated.

2.\ \ \ce{CsPbCl3} films fabricated by vacuum thermal-evaporation was employed in UV photodetectors. It was found that as the annealing temperature increased, the crystal size of \ce{CsPbCl3} increased and the crystallinity of films was also enhanced. The external quantum efficiency of UV photodetectors based on \ce{CsPbCl3} films increased from blow 5\% to around 20\%, which is in line with the improved morphology and the crystallinity. However, after annealing at higher temperature, the dark current density of the UV devices based on \ce{CsPbCl3} films deteriorated remarkable. It was attributed to the decreased interaction of cesium/acetate (which can effectively passivate traps of Cs atoms in inorganic \ce{CsPbCl3} perovskite films), with the increasement of annealing temperature. The study on thermal evaporation of \ce{CsPbCl3} films and UV photodetectors may also benefit other optoelectronic applications in the future.

\end{enabstract}
\vspace{1em}\par\vfill

%%%------ 英文关键词 ------- %%%
\enkeywords{{\bfseries\zihao{-4}ultraviolet photodetectors, wide bandgap perovskite, perovskite quantum dots, inorganic perovskite }}


