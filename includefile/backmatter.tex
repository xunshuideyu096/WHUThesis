% !Mode:: "TeX:UTF-8"

%%% 此部分内容:  (1) 致谢  (2) 武汉大学学位论文使用授权协议书(无需改动)

%%%%%%%%%%%%%%%%%%%%%%%
%%%------- 攻博(硕)期间科研成果 -------%%%
%%%%%%%%%%%%%%%%%%%%%%%
\reseachresult

\begin{enumerate}[{[1]}]
\item  \textbf{\underline{Lihao Cui}}, Jiali Peng, Wei Li, Yalun Xu, Meijuan Zheng, Qianqian Lin. Ultraviolet photodetectors based on \ce{CH3NH3PbCl3} perovskite quantum dots-doped poly(triarylamine)[J]. \textbf{Physica Status Solidi-Rapid Research Letters}, 2020, 14(4): 1900653.

\item Jiali Peng, \textbf{\underline{Lihao Cui}}, Ruiming Li, Yalun Xu, Li Jiang, Yuwei Li, Wei Li, Xiaoyu Tian, Qianqian Lin. Thick junction photodiodes based on crushed perovskite crystal/polymer composite films[J]. \textbf{Journal of Materials Chemistry C}, 2019, 7(7): 1859-1863.

\item Jiali Peng, Kuangkuang Ye, Yalun Xu, \textbf{\underline{Lihao Cui}}, Ruiming Li, Hao Peng, Qianqian Lin. X-ray detection based on crushed perovskite crystal/polymer composites[J]. \textbf{Sensors and Actuators A: Physical}, 2020, 312: 112132.

\item	Li Jiang, Yuwei Li, Jiali Peng, \textbf{\underline{Lihao Cui}}, Ruiming Li, Yalun Xu, Wei Li, Yanyan Li, Xiaoyu Tian, Qianqian Lin. Solution-processed \ce{AgBiS2} photodetectors from molecular precursors[J]. \textbf{Journal of Materials Chemistry C}, 2020, 8(7): 2436-2441.

\item Jiali Peng, Chelsea Q. Xia, Yalun Xu, Ruiming Li, \textbf{\underline{Lihao Cui}}, Jack K. Clegg, Laura M. Herz, Michael B. Johnston, Qianqian Lin. Crystallization of \ce{CsPbBr3} single crystals in water for X-ray detection[J]. \textbf{Nature Communications}, 2021, 12(1): 1531.

\item	Ruiming Li, Jiali Peng, Yalun Xu, Wei Li, \textbf{\underline{Lihao Cui}}, Yanyan Li, Qianqian Lin. Pseudohalide additives enhanced perovskite photodetectors[J]. \textbf{Advanced Optical Materials}, 2021, 9(2): 2001587.

\item	Wei Li, Yalun Xu, Jiali Peng, Ruiming Li, Jiannan Song, Huihuang Huang, \textbf{\underline{Lihao Cui}}, Qianqian Lin. Evaporated perovskite thick junctions for X-ray detection[J]. \textbf{ACS Applied Materials \& Interfaces}, 2021, 13(2): 2971-2978.

\item	Wei Li, Yalun Xu, Xianyi Meng, Zuo Xiao, Ruiming Li, Li Jiang, \textbf{\underline{Lihao Cui}}, Meijuan Zheng, Chang Liu, Liming Ding, Qianqian Lin. Visible to near-infrared photodetection based on ternary organic heterojunctions[J]. \textbf{Advanced Functional Materials}, 2019, 29(20): 1808948.

\item 林乾乾, {\bf\songti{\underline{崔立豪}}},一种钙钛矿量子点掺杂的有机紫外探测器及其制备方法:CN111009613A[P],2020.04.14。

\end{enumerate}
%%%%%%%%%%%%%%%%%%%%%%%
%%% --------------- 致谢 ------------- - %%%
%%%%%%%%%%%%%%%%%%%%%%%
\acknowledgement
%感谢你、感谢他、感谢大家。
%本论文是在我的导师林乾乾教授悉心指导和帮助下完成的,在此我由衷地表示感谢!林老师作为我科研道路上的领路人和启蒙者,专业知识雄厚、治学态度严谨,让我在科学研究探索道路上受益匪浅。同时,我有幸成为林老师第一届研究生,在我读硕期间跟随林老师搭建实验室,在林老师手把手指导下做实验、写论文、写专利等,让我学到了很多知识和技能。
{\songti\zihao{-4}三年研究生时光匆匆而过,不知是研究生阶段的忙碌生活还是突如其来的新冠疫情加快了时间的流逝。细想来我与武大结缘已七年有余,似乎冥冥之中注定我就要在这里走一遭。

本论文的完成,离不开我的导师林乾乾教授的悉心指导和帮助,在此我由衷地表示感谢!林老师思路开阔、作风勤勉、态度严谨,为我们创造了良好的实验条件和研究氛围。读硕期间跟随林老师搭建实验室,在林老师手把手指导下做实验、写论文、写专利等等,让我不仅丰富了知识还提高了动手能力。在李哲师兄的介绍下,我有幸成为林老师的第一届研究生,在此也表示对李哲师兄的感谢!
%感谢桂鹏彬师兄、姚方师兄,是他们帮助我尽快熟悉实验室环境,引导我开展实验。
%三年研究生时光匆匆而过,不知是研究生阶段的忙碌生活还是突如其来的新冠疫情加快了时间的流逝。细想来我与武大结缘已七年有余,当年我在高中电脑机房选择了武汉大学物理科学与技术学院,就注定我要在这里走一遭。

%在大四上学期时我就找到了林乾乾老师,林老师是我们学院新进教师,我也有幸成为林老师的第一届研究生,随后跟随林老师搭建实验室,在林老师手把手指导下做实验、写论文、写专利等等,让我学习到了很多知识和技能。研究生阶段的顺利度过和本论文的完成,我首先要感谢林老师指导与帮助。我还要感谢我的父母,虽然父母文化水平不高、家里生活条件也不是很好,但是父母还是一如既往地支持我读书求学。我还要感谢李睿明、李威和许亚伦,从东湖新村出租屋

本论文使用~\LaTeX~书写,要感谢武汉大学黄正华老师提供的~\LaTeX~使用教程和硕士研究生毕业论文模板,感谢本科室友李云飞同学提供的学习资源,还要感谢清华大学李泽平在参考文献格式方面提供的指导,他们的帮助使我极大地提高了论文书写效率。

感谢方国家老师课题组在实验条件上所提供的支持,并感谢方老师课题组的桂鹏彬师兄和姚方师兄,是他们帮助我尽快熟悉实验室环境,引导我顺利开展实验。还要感谢我们学院的刘雍老师、蒲十周老师在实验测试分析方面提供的指导。

三年研究生时光短暂而快乐,离不开同课题组兄弟姐妹的陪伴与支持。感谢和我同级的李睿明、李威和许亚伦,他们从不吝啬分享生活中的趣事和科研上的心得;感谢彭家丽师姐在测试分析上提供的热心帮助;此外,还有江力师兄,李雨薇、李妍妍、郑美娟、田晓语等师姐,黄辉煌、齐一鸣、余甜、桂福兵、宋建楠、喻岚鑫、白颂雪等师弟师妹们,感谢他们一同创造的融洽氛围,让我度过三年快乐的时光。

%感谢我的朋友,长久以来一直激励我勇敢前行 

感谢我的父母和亲人,还有小璐同志,感谢他们一如既往的支持和关心,为我在求学路上提供最坚实的依靠。

感谢各位专家、老师,感谢他们在百忙之中参与审阅和答辩。}

%%%%%%%%%%%%%%%%%%%%%%%%%%%%%%%%%%%%%%%
%%%%%%%--判断是否需要空白页-----------------------------
  \iflib
  \else
  \newpage
  \cleardoublepage
  \fi
%%%%%%%-------------------------------------------------







