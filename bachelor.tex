
% !Mode:: "TeX:UTF-8"
%% 请使用 XeLaTeX 编译本文.
\documentclass[forlib]{WHUMaster}   % 选项 forprint: 交付打印时, 建议加上此选项, 以消除彩色链接文字, 避免彩色字迹打印偏淡.
                                              % 选项 forlib: 提交给图书馆的电子版, 需要加上选项 forlib, 以消除空白页和彩色链接.
                                              % 选项 smd: Specialist Master's Degree, 产生专业硕士学位论文封面、页眉.
%%%=== 参考文献 === %%%
\usepackage{gbt7714}          %GBT7714-2005无法识别网页,故采用https://github.com/CTeX-org/gbt7714-bibtex-style提供的.bst文件,需要调用宏包gbt7714,2021/2/26 Lihao Cui
\bibliographystyle{gbt7714-numerical}        % 此参考文献格式在BIBbase中,需要将.bst文件放入到本地texmf目录中,参考https://github.com/Haixing-Hu/GBT7714-2005-BibTeX-Style,2020/12/10 Lihao Cui%使用"t" change.case$ 'lastname :=  将作者姓名改为Cui L此种格式,将格式另存后使用cmd命令texhash更新一下,崔立豪,20210329
%为实现作者Lin Q这种显示效果,可将.bst文件中的#1 'uppercase.name := 改成 #0 'uppercase.name :=,改完后保存再次运行即可,无需重命名文件再次调用,重命名调用可能会出现意料之外的问题,崔立豪,20210330

\def\degree{${}^{\circ}$} %定义角度°的输入,Lihao Cui
\usepackage{siunitx} %使用siunitx宏包来表示科学计数法,2021/1/20 Lihao Cui%也可使用\si{\um},\si{\uL}输入μm、μL,2020/3/23,clh;直接输入μ也可行,崔立豪
\usepackage[version=4]{mhchem} %使用mhchem宏包,来方便化学式的书写,2021/1/21 Lihao Cui
\usepackage{textcomp}
\usepackage{upgreek}
\usepackage{unicode-math} %调用unicode-math宏包方便正体希腊字母书写,使用\upmu可得到正体的希腊字母μ,20201/1/21 Lihao Cui 
\usepackage{float}%强制图片位置

\usepackage{fontspec}%[no-math]
\setmainfont{Times New Roman}%设置Times New Roman字体

\usepackage{unicode-math}

\setmathfont{TeX Gyre Termes Math}
%\setmathfont{STIX Two Math}%设置公式字体为接近Times New Roman的字体,崔立豪,20210427

\hypersetup{hidelinks}%消除超链接,崔立豪,20210317
%\setmainfont{\songti\zihao{-4}}

%\usepackage{xeCJK}
%\xeCJKsetup{AutoFakeSlant={true},AutoFakeBold={true}}
\setlength{\headheight}{13pt}

\usepackage{colortbl}
\usepackage{arydshln}
\usepackage{multirow}
\usepackage{multicol}%使用\cdashline绘制封面处的虚线,使用\cdashline{2-2}[2pt/0.4pt]\cdashline{5-5}[2pt/0.4pt]可达到和学校模板类似效果,2pt表示黑线长度,0.4pt表示空白长度,崔立豪,20210324

\captionsetup{labelformat=default,labelsep=space} %去除图注冒号,崔立豪,20210329
%\usepackage{natbib}
\setlength{\bibsep}{0.5ex}%减小参考文献之间的间距,崔立豪,20210329

%\usepackage[subfigure]{tocloft} 

\usepackage{titletoc}
\titlecontents{chapter}[4.5em]{\bfseries\zihao{-4}}{\contentslabel{4em}}{\hspace*{-4em}}{~\titlerule*[0.6pc]{$.$}~\contentspage}
\titlecontents{section}[4.5em]{\songti\zihao{-4}}{\contentslabel{2em}}{\hspace*{-4em}}{~\titlerule*[0.6pc]{$.$}~\contentspage}
\titlecontents{subsection}[8em]{\songti\zihao{-4}}{\contentslabel{3em}}{\hspace*{-2em}}{~\titlerule*[0.6pc]{$.$}~\contentspage}
\titlecontents{subsubsection}[12em]{\songti\zihao{-4}}{\contentslabel{3.5em}}{\hspace*{-2em}}{~\titlerule*[0.6pc]{$.$}~\contentspage}
%修改目录格式,崔立豪,20210426


%%%%%%%%%%%%%%%%%%%%%%%%%%%%%%%%%%%%%%%%%%%%%%%%%%%%
\begin{document}
%%%%%%%-------------------------------------------------

\fenleihao{\centerline{TN3}}  % 分类号:《中国图书资料分类法》的类号. 必填. 要根据自己的学科方向填写!!%\centerline居中,崔立豪,20210318%TN3为半导体技术,崔立豪
\miji{}                % 密级
\UDC{}               %《国际十进制分类法UDC》的类号. 选填.
\bianhao{\centerline{10486}}  % 学校编号, 10486是武汉大学的编号. 不用改动.%\centerline居中,崔立豪,20210318

\title{宽带隙钙钛矿材料在紫外光电探测器中的应用研究}
\Etitle{\zihao{2}The wide bandgap perovskites based ultraviolet photodetectors} % 英文题目
\author{崔立豪}
\StudentNumber{\zihao{4}2018202020058}   % 学号
\Eauthor{\zihao{4}Lihao Cui}            %作者英文名
\Csupervisor{林乾乾\quad 教授}        %指导教师中文名、职称
\Esupervisor{\zihao{4}Prof.~Qianqian Lin }   %指导教师英文名、职称
\Cmajor{材料物理与化学}                          % 专业中文名[ SMD:专业类别(领域)]
\Emajor{\zihao{4}Materials Physics and Chemistry}% 专业英文名[ SMD:专业类别(领域)]
\Cspeciality{半导体材料与器件}                     % 研究方向
\Especiality{\zihao{4}Semiconductor Materials and Devices}  % 研究方向
\Schoolname{School of Physics and Technology} %学院英文名. 不确定的话, 请看一下自己学院的网页上是怎么写的. 别搞错了!
\date{二〇二一年四月}                    % 硕士类只写年月. 要注意和英文日期一致!!
\Edate{April, 2021}                   % 英文封面日期

%-----------------------------------------------------------------------------
\pdfbookmark[0]{封面}{title}         % 封面页加到 pdf 书签
\maketitle
%-----------------------------------------------------------------------------
\let\cleardoublepage\clearpage
\include{includefile/frontmatter}    % 加入英文封面
\frontmatter
\pagenumbering{Roman}               % 正文之前的页码用大写罗马字母编号.

\cleardoublepage
\newpage  \pagestyle{fancy} \fancyfancy
%------------------------------------------------------------------------------
% !Mode:: "TeX:UTF-8"

%%% 说明: 此部分需要自己填写的内容:  (1) 中文摘要及关键词 (2) 英文摘要及关键词

%%%%%%%%%%%%%%%%%%%%%%%
%%% ------------ 中文摘要 ---------------%%%
%%%%%%%%%%%%%%%%%%%%%%%
\begin{cnabstract}

紫外光电探测器因具有在民用与军事领域上的巨大应用前景而备受研究人员的关注。宽带隙无机物——尤其是可使用低成本简单工艺制备的氧化锌(ZnO)——是制作紫外光电探测器的理想半导体材料。但由于p型ZnO材料不稳定,基于ZnO同质结的紫外光电探测器较少被报道。而将ZnO和有机半导体复合使用,为解决p型ZnO材料不稳定的问题提供了一种替代性方案。但是有机半导体较低的载流子迁移率,一定程度上限制了器件性能的提高。

金属卤化物钙钛矿材料具有带隙可调、载流子迁移率较高等优异的光电性能,被广泛应用于制作高性能的光电器件。其中氯化物钙钛矿具有合适的带隙宽度,可用于紫外光电探测器。但是有关基于稳定性更好的全无机钙钛矿的器件的报道相对较少,这是由于全无机钙钛矿前驱体在常规有机溶剂中溶解度较低,难以通过成熟的溶液法来制备,阻碍了它在紫外光电探测器上的应用。

针对上述问题,本文以提高基于ZnO与有机半导体复合层的紫外光电探测器的器件性能为出发点,研究宽带隙全无机钙钛矿材料的制备工艺及其在紫外光电探测器方面的应用,进行了以下探索:

其一,通过过饱和溶液再结晶方法制备\ce{CH3NH3PbCl3}钙钛矿量子点,并将其掺杂到聚[双(4-苯基)(2,4,6-三甲基苯基)胺](PTAA)聚合物导电薄膜中。研究表明,向PTAA中加入适量钙钛矿量子点,可在不影响器件暗电流密度的情况下提高光电流密度,并提升器件的响应度和比探测度。此外,\ce{CH3NH3PbCl3}量子点的引入可提高PTAA薄膜的空穴载流子迁移率,更有利于电荷的传输和抽取,进而提升紫外探测器的光电性能。

其二,尝试将全无机钙钛矿用作紫外光电探测器的光吸收层,使用热蒸发方式制备\ce{CsPbCl3}薄膜。研究表明,在随着对\ce{CsPbCl3}薄膜退火温度的增加,薄膜的晶粒增大、结晶性变好,并且紫外光电探测器的外量子效率从不足5\%上升至20\%左右。而使用较高温度对\ce{CsPbCl3}钙钛矿薄膜退火处理后,ZnO薄膜中的醋酸根离子减少,使可抑制钙钛矿薄膜中缺陷的相互作用减弱,导致紫外光电探测器性能下降。这一研究为制作基于宽带隙全无机钙钛矿(如\ce{CsPbCl3})薄膜的光电器件奠定了一定基础。

\end{cnabstract}
\vspace{1em}\par\vfill

%%%--------- 关键词 -------- %%%
\cnkeywords{{\heiti\zihao{-4}紫外光电探测器,宽带隙钙钛矿材料,钙钛矿量子点, 无机钙钛矿} }

%%%%%%%%%%%%%%%%%%%%%%%


%%%%%%%%%%%%%%%%%%%%%%%
%%% ------------ 英文摘要 ---------------%%%
%%%%%%%%%%%%%%%%%%%%%%%

\begin{enabstract}%This thesis is a study on the applications of perovskite materials on ultraviolet photodetectors\dots.

Ultraviolet (UV) photodetectors have attracted much attention for their wide applications in civilian and military fields. Wide bandgap inorganic semiconductor{\textemdash}zinc oxide (ZnO) could be fabricated with low-cost methods, and have emerged as a promising candidate for UV detection and imaging. Nevertheless, ZnO based homojuction photodetectors have been rarely reported owing to the instability of p-type ZnO. Combing organic semiconductor materials with ZnO can effectively solve this issue. However, this type of photodetectors usually suffers from the relatively low responsivity, due to the inferior charge transport properties.

Metal halide perovskite have been widely used in many high-performance optoelectronic devices due to their brilliant properties, such as bandgap adjustability and large carrier mobility. Furthermore, chlorine based organic-inorganic hybrid perovskites with suitable band gap have been introduced for UV photodetection. However, efficient UV photodetectors based on inorganic perovskites have been barely reported due to their poor solubility.

To solve those problems, we incorporated the \ce{CH3NH3PbCl3} quantum dots (QDs) with high carrier mobility into thick organic semiconductor layer to enhance the film charge transport properties and device performance. Besides, we also investigated the possibility of replacing organic layer directly by inorganic perovskite as the active layer in UV photodetectors. The key elements of this thesis is as follows:

1.\ \ \ce{CH3NH3PbCl3} perovskite QDs prepared via supersaturated recrystallization were doped into Poly(triarylamine) (PTAA) organic semiconductor films. With proper doping amount of \ce{CH3NH3PbCl3} QDs into PTAA, the responsivity of the UV devices could be increased without distinct change of the dark current density, and the specific detectivity of photodetectors could be improved as well. Furthemore, the hole mobility of doped PTAA was determined to be enhanced, that may facilitate the charge transport, and result improved device performance. In addition, the great long-time stability and flexibility of these devices also be demonstrated.

2.\ \ \ce{CsPbCl3} films fabricated by vacuum thermal-evaporation was employed in UV photodetectors. It was found that as the annealing temperature increased, the crystal size of \ce{CsPbCl3} increased and the crystallinity of films was also enhanced. The external quantum efficiency of UV photodetectors based on \ce{CsPbCl3} films increased from blow 5\% to around 20\%, which is in line with the improved morphology and the crystallinity. However, after annealing at higher temperature, the dark current density of the UV devices based on \ce{CsPbCl3} films deteriorated remarkable. It was attributed to the decreased interaction of cesium/acetate (which can effectively passivate traps of Cs atoms in inorganic \ce{CsPbCl3} perovskite films), with the increasement of annealing temperature. The study on thermal evaporation of \ce{CsPbCl3} films and UV photodetectors may also benefit other optoelectronic applications in the future.

\end{enabstract}
\vspace{1em}\par\vfill

%%%------ 英文关键词 ------- %%%
\enkeywords{{\bfseries\zihao{-4}ultraviolet photodetectors, wide bandgap perovskite, perovskite quantum dots, inorganic perovskite }}


      % 加入中英文摘要.
%---把目录加入到书签---%%%%%%%%%%%%%%
\pdfbookmark[0]{目录}{toc}%%%%%%%%%%%%

\tableofcontents

%------------------------------------------------------------------------------
\mainmatter{\zihao{-4}} %% 以下是正文
\baselineskip=20pt  % 正文行距为 20 磅
%%%%%%%%%%%%%%%%%%%%%%%%%%%%%%%%%%%%
\chapter{绪论}

\section{引言}%简单总结本论文研究背景和内容。

{\songti\zihao{-4}紫外光是太阳辐射中的重要组成部分,与人类的生存和发展息息相关。适量的紫外光照射有利于身体健康,但如果受到过量紫外光照射则可导致皮肤癌、白内障等疾病,甚至会加速人体老化过程\cite{RN182}。随着半导体材料与器件在理论研究和制备工艺上的不断发展,可用于检测和测量紫外光辐射的半导体紫外光电探测器被创造出来。在过去的几十年里,紫外光电探测器因在通讯、臭氧空洞监测、火焰检测以及导弹追踪等民用和军事领域中的巨大应用前景而备受研究人员的关注\cite{RN182,RN15}。氮化镓(GaN)、碳化硅(SiC)以及氧化锌(ZnO)等诸多无机物是制作紫外光电探测器的理想半导体材料,尤其是ZnO,由于其制备成本较低而被广泛用于紫外光检测和成像。不过,很少有基于ZnO同质结的光电二极管型器件的研究报道,这是因为p型ZnO材料存在着稳定性较差的问题。如果选择合适的p型有机半导体与ZnO复合使用,可以有效解决这一问题,并且基于复合材料的器件兼具无机和有机半导体的优势,进一步拓宽了ZnO在紫外探测领域的应用。比如,Esopi等人使用p型有机半导体聚[双(4-苯基)(2,4,6-三甲基苯基)胺](PTAA)和ZnO纳米颗粒的混合物制作了具有窄带响应的有机无机杂化型紫外光电探测器\cite{RN36}。然而,有机半导体的载流子迁移率一般都比较低,例如,Intaniwet等人采用飞行时间法测得PTAA的空穴载流子迁移率只有\num{1.3e-6}\ \si{cm^2\ V^{-1}\ s^{-1}}\cite{RN34},这在一定程度上限制了器件性能的提高,从而阻碍了基于ZnO/有机半导体复合材料的紫外光电探测器的发展。

金属卤化物钙钛矿材料具有带隙可调、载流子迁移率较高、吸光系数较高以及载流子寿命较长等优异的光电特性,且制备工艺简单、制作方法多样、材料种类和形态丰富,在太阳能电池、发光二极管、光电探测器等领域均有广泛的应用研究。其中,在紫外光电检测方面,宽带隙钙钛矿材料亦有用于制作高性能紫外光电探测器的巨大潜力。此外,相较于有机无机杂化钙钛矿材料,全无机钙钛矿材料具有更好的稳定性。不过,用于制备全无机钙钛矿薄膜的前驱体在常规有机溶剂中的溶解度较低,这在一定程度上限制了宽带隙全无机钙钛矿材料在紫外光电探测器中的应用。

本文以紫外光电探测器为研究主体,从提高应用于紫外光电探测器的有机半导体载流子迁移率和寻找不受溶解度限制的制备全无机钙钛矿薄膜方法出发,以提高紫外光电探测器的性能和拓宽钙钛矿材料在紫外光电探测器的应用为目标,开展了许多探索研究工作。先是为降低有机半导体载流子迁移率较低对器件性能的影响,尝试将\ce{CH3NH3PbCl3}(\ce{MAPbCl3})钙钛矿量子点掺杂到有机半导体中。\ce{MAPbCl3}钙钛矿量子点材料的引入使器件各层之间能级更加匹配,并且提高了有机半导体层的载流子输运性能,进而提高了紫外光电探测器的响应度、比探测度、响应速度等关键器件性能参数。此外,还验证了同种器件结构应用于柔性紫外光电探测器的可行性。再者,考虑到有机无机杂化钙钛矿较差的稳定性,还进一步制作了以全无机钙钛矿材料为光吸收层的紫外光电探测器。针对使用常规溶液法难以制备宽带隙全无机钙钛矿薄膜的问题,使用了热蒸发的方法来沉积全无机\ce{CsPbCl3}钙钛矿薄膜,系统探索了\ce{CsPbCl3}薄膜的制备工艺,并将其应用到器件结构较简单的紫外光电探测器中,研究了不同退火温度处理对\ce{CsPbCl3}薄膜形貌和基于此的紫外光电探测器性能的影响。}

\section{钙钛矿材料特性及其制备工艺简介}

{\songti\zihao{-4}钙钛矿(perovskite)这个名字源于1839年德国矿物学家Gustav Rose发现的一种矿物钛酸钙(\ce{CaTiO3})的结构,后来这种结构被俄国矿物学家Lev A. Perovski进一步研究表征,并由此衍生出它的名字\cite{RN52}。本论文研究的卤化物钙钛矿材料具有和\ce{CaTiO3}类似的结构,如无特别说明,后续提及的钙钛矿均为卤化物钙钛矿。}

\subsection{钙钛矿材料的组成与结构}%钙钛矿材料的结构与优良光电特性(重点提及带隙可调)%钙钛矿材料的吸收系数、载流子迁移率、载流子寿命等

{\songti\zihao{-4}对于三维卤化物钙钛矿材料,可用化学式\ce{ABX3}表示,其X位离子可以是氟离子(\ce{F^-})、氯离子(\ce{Cl^-})、溴离子(\ce{Br^-})、碘离子(\ce{I^-})等一种或多种卤素离子,A位离子可以是铯离子(\ce{Cs^+})、甲胺根离子(\ce{CH3NH3^+},\ce{MA^+})以及甲脒根离子(\ce{CH(NH2)2^+},\ce{FA^+})等一种或多种一价阳离子,B位离子可以是锡离子(\ce{Sn^{2+}})、铅离子(\ce{Pb^{2+}})等一种或多种二价阳离子\cite{RN118,RN137}。时至今日,\ce{MAPbI3}\cite{RN139,RN122,RN123}、\ce{MAPbBr3}\cite{RN124,RN125}、\ce{MAPbCl3}\cite{RN126,RN127,RN128}、\ce{MAPbI_{3-x}Cl_x}\cite{RN129}、\ce{MAPbBr_{3-x}Cl_x}和\ce{MAPbBr_{3-x}I_x}\cite{RN61,RN130,RN131,RN132}等多种钙钛矿材料已被广泛应用于光电器件。使用\ce{FA^+}替换掉A位的\ce{MA^+}离子,可以拓宽材料光谱吸收范围,并能在一定程度上提高材料的稳定性\cite{RN133}。如果进一步将A位的有机离子(比如,\ce{MA^+}和\ce{FA^+})替换为无机离子(比如,\ce{Cs^+}),则可以显著提高材料的热稳定性\cite{RN134}。}



\begin{figure}[htb]
\centering
  \includegraphics[width=0.95\textwidth]{Fig_1.2.1.jpg}
  \caption{\rm \ (a) 三维钙钛矿晶体结构示意图,(b) 不同钙钛矿的容忍因子$t$以及八面体因子$\mu$\cite{RN136},(c)层状钙钛矿材料晶体结构随着$n$的数值变化的演变过程\cite{RN135}。}
  \label{fig:1.2.1}
\end{figure}

{\songti\zihao{-4}三维钙钛矿材料具有图\ref{fig:1.2.1}a所示的晶体结构,一般A位离子半径较B位和C位离子半径大\cite{RN52}。其中A位离子占据晶格角落,B位离子占据晶格间隙,X位离子位于晶面上。钙钛矿的电学性质主要由构成无机框架的B-X键决定,A位离子不直接对其电学性质产生影响,但离子半径不同大小的A位离子可使B-X键发生不同程度的形变,从而间接影响钙钛矿的性质\cite{RN52,RN137}。在理想情况下,图\ref{fig:1.2.1}a所示的这种结构为立方对称结构\cite{RN52,RN55},改变离子半径的大小可使晶体结构发生变形。通常可用容忍因子$t$(Goldschmidt's tolerance factor)来描述这种因离子半径大小引起的对立方对称结构的偏移程度\cite{RN54},容忍因子$t$可表述为:

%钙钛矿材料结构的稳定性和可能存在的钙钛矿结构可以用Goldschmidt's tolerance factor(容忍因子)$t$和octahedral factor(八面体因子)$\mu$来预测。在20世纪20年代初Goldschmidt就提出了容忍因子$t$\iffalse \cite{Formability of ABX3 (X = F, Cl, Br, I) halide perovskites} \fi:

\begin{equation}\label{eq-1.1}
{t = \frac{r_{\rm{A}}+r_{\rm{X}}}{\sqrt{2}(r_{\rm{B}}+r_{\rm{X}})}}
\end{equation}

其中,$r_{\rm{A}}$、$r_{\rm{B}}$和$r_{\rm{X}}$分别代表A、B和X位离子的离子半径。对于卤化物钙钛矿材料,要形成立方结构,一般要求0.85<$t$<1.11\cite{RN55},当$t = 1$时,钙钛矿具有完美的立方结构\cite{RN137}。此外为更充分评估钙钛矿材料的结构,还需引入另外一个与\ce{BX6}八面体相关的参数——八面体因子$\mu$,该因子被定义为B位离子半径和X位离子半径之比:

\begin{equation}\label{eq-1.2}
\mu = \frac{r_{\rm{B}}}{r_{\rm{X}}}
\end{equation}

对于卤化物钙钛矿,一般要求0.44<$\mu$<0.90\cite{RN56},在这个范围之外\ce{BX6}八面体将变得不稳定,无法形成三维钙钛矿结构。不过,虽然容忍因子$t$和八面体因子$\mu$为卤化物钙钛矿形成可能性给出了合理的指导,但是它们并不能完全预测钙钛矿家族中所有成员的形成\cite{RN55}。图\ref{fig:1.2.1}b展示了一些常见卤化物钙钛矿材料的$t$和$\mu$的数值。使用X射线衍射(XRD)分析表面钙钛矿晶体,通常存在立方\iffalse(Pm3m)\fi、四方\iffalse(I4/mcm)\fi 和正交三种相\cite{RN57}。一般情况下,钙钛矿都是立方相,随着温度的降低钙钛矿材料会发生由立方相到四方相再到正交相的转变。\cite{RN52}%另外,对于全无机钙钛矿\ce{CsPbI3}、\ce{CsPbBr3}和\ce{CsPbCl3}其容忍因子$t$分别是0.851、0.862和0.870,因为材料的$t$值越接近1就越稳定,说明\ce{CsPbCl3}在三种材料中最为稳定。\iffalse \cite{cesium lead halide (CsPbX3, x= I, Br, Cl) perovskites} \fi

和三维结构钙钛矿材料不同,层状结构钙钛矿材料一般可用\ce{(RNH3)2A_{n-1}B_nX_{3n+1}}来表示,其中${\rm{R}}$表示长链的烷基或芳香族,$n$表示两层有机离子链之间金属离子层的层数。层状钙钛矿材料一般是由原子层厚的金属卤化物薄片堆积而成,这些薄片被长链有机分子分开,相邻层之间通过弱范德华力连接。层状钙钛矿材料的厚度和光电特性可通过改变$n$的数值进行调节,当$n$ = 1时,此时层状钙钛矿材料为纯二维钙钛矿,化学式为\ce{(RNH3)2BX_{4}};当$n$ = $\infty$时,此时材料就是三维钙钛矿;当$n$ = 其他整数时,此时钙钛矿材料可被称为准二维材料。图\ref{fig:1.2.1}c展示了层状钙钛矿材料晶体结构随着$n$的数值变化的演变过程。由于二维和准二维钙钛矿材料相比三维钙钛矿材料具有更好的湿度稳定性,因此将之应用于光电器件的研究越来越多\cite{RN135}。}

\subsection{钙钛矿的光电特性}

{\songti\zihao{-4}钙钛矿是一种直接带隙半导体材料\cite{RN59},它的带隙可通过调控X位、A位以及B位的离子组成来进行调节。如同图\ref{fig:1.2.2}a所示,通过改变组成\ce{CsPbX3}钙钛矿X位离子的组分及比例,可以得到在紫外灯照射下呈现不同颜色的钙钛矿纳米晶体胶体分散液,其对应于如图\ref{fig:1.2.2}b所示的光致发光(photoluminescence,PL)光谱,说明\ce{CsPbX3}钙钛材料的带隙可以在1.77 eV(\ce{CsPbI3})至3.06 eV(\ce{CsPbCl3})之间调节\cite{RN60}。类似地,又如图\ref{fig:1.2.2}c所示,通过\ce{MAPbX3}钙钛矿X位离子的组分及比例的变化,可以得到不同颜色的钙钛矿纳米晶体胶体分散液,其对应的紫外-可见漫反射光谱和PL光谱(图\ref{fig:1.2.2}d)覆盖了400-850 nm波长范围,表明\ce{MAPbX3}钙钛矿材料的带隙可在1.5 eV至3.1 eV之间变化\cite{RN61}。从上面两个例子也可总结出,通过改变钙钛矿A位离子的成分也可对钙钛矿材料的带隙进行调节。当然,使用\ce{Sn^{2+}}将B位的\ce{Pb^{2+}}进行不同比例的替换也可对钙钛矿材料的带隙进行调控\cite{RN69}。

\begin{figure}[ht]
\centering
  \includegraphics[width=0.95\textwidth]{Fig_1.2.2.jpg}
  \caption{\rm \ (a) \ce{CsPbX3} (X = Cl,Br,I)钙钛矿纳米晶体的胶体分散液在紫外灯照射下的图片和(b) 相应钙钛矿的光致发光(photoluminescence,PL)光谱\cite{RN60},(c) \ce{MAPbBr_{3-x}Cl_x}和\ce{MAPbBr_{3-x}I_x}钙钛矿纳米晶体胶体分散液图片和(d)相应钙钛矿纳米晶体薄膜的紫外-可见漫反射光谱和PL光谱\cite{RN61}。}
  \label{fig:1.2.2}
\end{figure}

Tao等人系统研究了钙钛矿带隙随X位、A位以及B位的离子变化的规律\cite{RN137}。通过分析卤化物钙钛矿材料的电子结构,发现钙钛矿材料的能级主要由金属原子(M)和卤素原子(X)的$s$和$p$轨道杂化构成,具体来说,导带(VB)顶部主要由M原子的$s$轨道(M,$s$)和X原子的$p$轨道(X,$p$)决定,价带(CB)底部则主要由M原子的$p$轨道(M,$p$)和X原子的$s$轨道(X,$s$)决定,并且M,$p$对导带最低能级(CBM)起决定性作用,进一步使用紧束缚分析就到了如图\ref{fig:1.2.2.2}所示不同钙钛矿材料的能级图。

\begin{figure}[ht]
\centering
  \includegraphics[width=0.95\textwidth]{Fig_1.2.2.2.jpg}
  \caption{\rm \ \ce{AMX3}结构钙钛矿的能级示意图\cite{RN137}。 (a) (b) 分别为改变卤素离子和金属离子对钙钛矿材料能级影响示意图,(c)改变锡基钙钛矿A位离子时能级变化示意图,(d)按照材料带隙减小顺序排列的18种卤化物钙钛矿能级示意图。}
  \label{fig:1.2.2.2}
\end{figure}

首先是图\ref{fig:1.2.2.2}a所示的改变钙钛矿X位离子对材料带隙的影响。CBM的能量主要受${\rm{Pb}},p$原子能级位置的影响,其能量随着卤素从$\rm{I}$到$\rm{Br}$再到$\rm{Cl}$变化而上升,这是因为随着Pb-X之间距离的减小,Pb原子上的电子由于受到更强的束缚而能量增加;价带最高能级(VCM)主要受X,$p$的影响,卤素从$\rm{I}$到$\rm{Br}$再到$\rm{Cl}$变化电负性逐渐增加,从而使得VCM出现下降趋势。因此,随着卤素从$\rm{I}$到$\rm{Br}$再到$\rm{Cl}$变化,材料的CBM上移而VCM下移,最终使得带隙变大。再者是图\ref{fig:1.2.2.2}b所示的改变钙钛矿M位离子对带隙的影响。由于Sn较Pb具有更弱的电负性,使用Sn替换掉Pb,其原子能级会出现上升的趋势;又Sn原子$s$与$p$之间分裂程度小于Pb原子$s$与$p$之间的分裂程度,因此用Sn替换掉Pb后,材料的VBM比CBM上升程度更大,从而使材料的带隙变小。但有些钙钛矿(\ce{FASnBr3}、\ce{MASnCl3}和\ce{FASnCl3})并不符合这条规律,说明还需考虑其他影响因素,如,晶格形变等。如果只考虑替换A位的离子(图\ref{fig:1.2.2.2}c),由于A位离子不直接参与成键,其对材料的电学性质贡献可以从对晶体结构的影响上来进行分析。钙钛矿结构的变形,会在一定程度上减弱M和X原子的杂化,使材料的VCM和CBM发生下移,而由于VBM对杂化更敏感,其受到的影响也更大;当A位离子从Cs到MA再到FA离子半径逐渐增大而导致晶胞体积增大,这会使M能级出现下降,从而导致材料的VCM和CBM下移,但CBM下降的程度要更大一些。总之,晶体结构形变和晶胞体积增加都会使VCM和CBM下移,前者使VBM下降较多,后者使CBM下降较多,综合考虑两种影响,可以给带隙变化带来更合理的解释。对于\ce{ASnCl3}钙钛矿,A位离子从Cs到MA再到FA离子变化时晶格形变程度逐渐增加,这会使VCM和CBM下移,并且CBM下移程度会大一些,这就导致了材料的带隙逐渐增加。\iffalse 显然,可以同时调节A位、B位以及X位的离子组分来制备合适带隙的钙钛矿材料,材料带隙调节的灵活性大大增加了钙钛矿材料的应用范围。\fi

钙钛矿还具有较高的光吸收系数,将其应用于光电器件中可显著减少材料所需的厚度\cite{RN52,RN47}。此外钙钛矿材料还具有较长的载流子寿命\cite{RN48}以及较高的载流子迁移率\cite{RN49,RN50}等优异的光电性能,这些优异的光电特性为将其应用到高性能光电器件中打下了坚实的基础。}

%\subsection{钙钛矿材料在光电器件中的应用}%举例钙钛矿在太阳能电池、发光二极管、光电探测器中的应用。

%组成钙钛矿材料的原料种类丰富,并且这种材料还具有优异的光学及电学性质,为钙钛矿广泛应用开辟了道路。\iffalse 在2006年,Miyasaka和同事将\ce{CH3NH3PbBr3}钙钛矿材料作为纳米多孔\ce{TiO2}上敏化层应用于染料敏化太阳能电池(DSSC)并实现了2.2\%的能量转化效率\cite{RN53},这是有机-无机杂化钙钛矿材料在光伏器件中的首次亮相。在钙钛矿材料在DSSC中得以应用后,这种材料在光伏领域的应用吸引了大量研究人员的研究兴趣,时至今日钙钛矿太阳能电池的能量转换效率已高达25.5\%\cite{RN70}。在DSSC中使用钙钛矿材料后,人们很快就意识到钙钛矿材料是一种独特的半导体材料,不同于染料分子以及其他有机吸收材料,更像应用于光伏的无机半导体(比如,Si或GaAs)。\cite{RN71}\fi

\subsection{钙钛矿材料制备工艺简介}

{\songti\zihao{-4}钙钛矿材料在光电器件中的广泛应用离不开钙钛矿材料制备工艺的不断发展。用于光电器件的钙钛矿材料有钙钛矿多晶薄膜、钙钛矿量子点、钙钛矿纳米线、钙钛矿单晶薄膜及块体以及钙钛矿与其他材料的复合物等多种形态。下面将简单介绍本论文后续实验中所涉及的钙钛矿材料的制作工艺,分别是钙钛矿量子点的制备工艺和钙钛矿多晶薄膜的制备工艺。

\subsubsection{钙钛矿量子点制备工艺简介}%钙钛矿量子点制备工艺(热注入法、过饱和溶液结晶法等)

制备钙钛矿量子点主要有如图\ref{fig:1.2.3.1}所示的三种方法,分别是配体辅助再沉淀(ligand-assistted reprecipitation,LARP)(图\ref{fig:1.2.3.1}a-b)、热注入法(图\ref{fig:1.2.3.1}c)和介孔\ce{SiO2}模板法(图\ref{fig:1.2.3.1}d)。

\begin{figure}[ht]
\centering
  \includegraphics[width=0.9\textwidth]{Fig_1.2.3.1.jpg}
  \caption{\rm \ 制备钙钛矿量子点的工艺:(a)(b) 过饱和溶液结晶法\cite{RN10,RN76}、(c) 热注入法\cite{RN77}、(d) 介孔\ce{SiO2}模板法示意图\cite{RN80}。}
  \label{fig:1.2.3.1}
\end{figure}

\textbf{LARP法:}

LARP方法主要原理为过饱和溶液重结晶。所谓过饱和溶液重结晶即为当过饱和溶液处于非平衡状态时,使用搅拌或加入杂质等扰动,可使溶液中的离子以形成晶体的方式析出,这个自发沉淀和结晶的过程会一直持续直至系统再次达到平衡状态\cite{RN76}。使用LARP法制备钙钛矿量子点是利用了钙钛矿前驱体在不同溶剂中的溶解度不同的原理,先将钙钛矿溶解在溶解度较高的溶液里,然后将钙钛矿前驱体溶液加入到快速搅拌的对钙钛矿溶解度较低的溶液中,钙钛矿量子点就会快速形成,加入合适的有机配体可以控制量子点的尺寸和形貌并让量子点稳定分散在有机溶剂中\cite{RN76,RN33}。Zhang等人将LARP方法应用于合成\ce{MAPbX3}钙钛矿量子点(图\ref{fig:1.2.3.1}a),将混合有钙钛矿前驱体和长链有机配体(正辛胺、油酸等)的N,N-二甲基甲酰胺(DMF)溶液加入到快速搅拌的对钙钛矿溶解度较低的有机溶剂(甲苯、己烷等)中,即可获得钙钛矿量子点\cite{RN10}。Li等人后续将此种方法应用到\ce{CsPbX3}全无机钙钛矿量子点的合成中\cite{RN76}。

\textbf{热注入法:}

Protesescu等人发展了合成钙钛矿量子点胶体的热注入方法,这种方法一般需要140-200{\textcelsius}的温度\cite{RN60},图\ref{fig:1.2.3.1}c展示了这种方法示意图。首先让\ce{Cs2CO3}和油酸反应生成油酸铯盐,再将油酸铯盐快速注入到加热中的\ce{PbX2}溶液中,\ce{PbX2}和油酸铯盐在140-200{\textcelsius}高温下反应结晶形成\ce{CsPbX3}量子点,长链的有机配体(油胺、油酸等)被用于限定结晶晶粒的大小。这种方法合成的钙钛矿量子点的尺寸主要依赖于反应温度,与反应时间无关。因此,通过调节反应温度,晶粒的大小可以被精确调控。

\textbf{介孔\ce{SiO2}模板法:}

将介孔\ce{SiO2}作为模板,使用浓缩钙钛矿前驱体溶液浸泡介孔\ce{SiO2}材料,为防止在\ce{SiO2}微孔外形成钙钛矿晶体,通常使用滤纸或真空过滤将多余的钙钛矿溶液去除,最后在真空干燥箱中加热即可得到钙钛矿量子点\cite{RN80}。\iffalse 再将具有孔径为2.5-7 nm的六边形有序排列的一维通道或者孔径为15-50 nm的无序多空网络结构是制作量子点材料的良好模板。\fi 不同于LARP和热注入法,介孔\ce{SiO2}模板法合成的量子点不需要长链配体、不需要使用溶剂分散为胶体,因此可使用介孔\ce{SiO2}模板法大批量制备钙钛矿量子点。

\subsubsection{钙钛矿多晶薄膜制备工艺简介}%钙钛矿薄膜的制备工艺(一步反溶剂辅助法、两步法、热蒸发法等)

制备钙钛矿多晶薄膜的工艺有很多,可大致将其归类为“一步法”和“两步法”。“一步法”和“两步法”的区别主要在于钙钛矿薄膜是由前驱体混合溶液直接形成还是由先沉积无机薄膜再加入其他成分后转变而来的\cite{RN96,RN97,RN98}。图\ref{fig:1.2.3.2}展示了几种常见的制备钙钛矿薄膜的工艺示意图。

\begin{figure}[ht]
\centering
  \includegraphics[width=0.9\textwidth]{Fig_1.2.3.2-1.jpg}
  \caption{\rm \ 常见制备钙钛矿薄膜的“一步法”工艺包括:(a) 非化学计量前驱体沉积法、(b)(c) 在旋涂钙钛矿混合溶液时加入反溶剂以促进结晶、(e) 钙钛矿前驱体混合溶液与溶剂形成配位化合物和(f)溶剂退火后处理工艺。常见制备钙钛矿薄膜的“两步法”工艺包括:(d)先形成\ce{PbI2}和有机溶剂配位化合物薄膜或\ce{PbI2}薄膜再使用(g) 浸泡、(h) 旋涂、(i) 有机蒸汽等扩散方式形成钙钛矿薄膜\cite{RN95}。}
  \label{fig:1.2.3.2}
\end{figure}

常见制备钙钛矿薄膜的“一步法”工艺包括非化学计量前驱体沉积法、反溶剂辅助法、前驱体混合溶液与溶剂形成配位化合物法、溶剂退火后处理法等溶液法以及物理气相沉积法\cite{RN95}。对于“一步法”中的溶液法,早期的研究主要集中在通过调节前驱体的比例让钙钛矿薄膜尽可能覆盖衬底表面,如Lee等人将\ce{CH3NH3I}和\ce{PbCl2}以3:1的摩尔比例在N,N-二甲基甲酰胺(DMF)中混合,使基于钙钛矿薄膜的太阳能电池能量转换效率超过了10\%\cite{RN96}。此后,研究人员还使用了二甲基亚砜(DMSO)和N-甲基吡咯烷酮(NMP)等有机溶剂来溶解钙钛矿前驱体,并制备了具有不错性能的器件\cite{RN103}。Jeon等人和Xiao等人几乎同时报道了旋涂钙钛矿薄膜时在湿润的前驱体薄膜上滴加反溶剂可以加速薄膜结晶的方法\cite{RN100,RN104}。Xiao等人将\ce{CH3NH3I}和\ce{PbI2}以1:1的摩尔比例溶解在DMF中,并尝试了诸如氯苯、苯、二甲苯、甲苯、异丙醇和氯仿等反溶剂以加速薄膜结晶过程,使制备的薄膜中的晶粒大小达到了微米尺寸\cite{RN100}。另一方面,Jeon等人将钙钛矿前驱体溶解在$\gamma$-丁内酯和DMSO的混合溶液中,并使用了甲苯作为反溶剂\cite{RN104}。DMSO是制备钙钛矿薄膜常用到的溶剂,它会使\ce{PbI2}在薄膜顺序沉积的第一步中形成非晶薄膜,并且\ce{PbI2}在DMSO中的溶解度(>2 M)大于在DMF中的溶解度,这主要是与\ce{Pb^{2+}}和溶剂中带负电的原子(比如氧原子)所形成的配位键强弱有关\cite{RN105}。含铅前驱体和DMSO分子之间较强的配位键还会减慢钙钛矿相的转变进程,进而使得到的钙钛矿薄膜具有更均匀、尺寸更大的晶粒\cite{RN105,RN106,RN109,RN110}。溶剂对钙钛矿薄膜的形貌有很大影响,Liu等人发展了溶剂退火后处理工艺(图\ref{fig:1.2.3.2}f) ,提高了钙钛矿薄膜的结晶性\cite{RN106}。

\begin{figure}[ht]
\centering
  \includegraphics[width=0.95\textwidth]{Fig_1.2.3.2.2.jpg}
  \caption{\rm \ (a) 使用热蒸发法沉积钙钛矿薄膜示意图,使用(b) 热蒸发法和 (c) 溶液法沉积得到的钙钛矿薄膜的SEM图片\cite{RN99}。}
  \label{fig:1.2.3.2.2}
\end{figure}

除了上述介绍的“一步法”工艺,还可以使用“两步法”工艺来制备钙钛矿薄膜。Burschka等人报道了一种将\ce{PbI2}薄膜浸泡在含有\ce{MAI}的异丙醇(IPA)溶液中的“两步法”工艺来制备钙钛矿薄膜\cite{RN97},其原理如图\ref{fig:1.2.3.2}g所示。首先使用\ce{PbI2}的DMF溶液在介孔\ce{TiO2}薄膜上旋涂制得\ce{PbI2}薄膜,然后将\ce{PbI2}薄膜浸泡到含有\ce{MAI}的IPA溶液中,最后使用IPA进行冲洗后退火即可得到\ce{MAPbI3}钙钛矿薄膜。其中IPA对已形成的\ce{PbI2}薄膜影响较小,因为\ce{PbI2}在IPA中的溶解度较低。\ce{MAPbI3}钙钛矿薄膜的形成是\ce{MAI}和\ce{PbI2}相互反应的结果,在此种方法中,\ce{MAI}不仅和表面处的\ce{PbI2}反应还会随着浸泡时间的延长和薄膜内部的\ce{PbI2}发生反应,因此这种方法的关键参数就是\ce{PbI2}薄膜在含有\ce{MAI}的IPA溶液中的浸泡时间。使用扫描电子显微镜照片(SEM)观察薄膜截面,可以看出\ce{PbI2}完全渗透进入介孔\ce{TiO2}薄膜内,介孔\ce{TiO2}薄膜将\ce{PbI2}的晶粒大小被控制在\~{} 22 nm,从而提高了\ce{MAI}和\ce{PbI2}的反应速度,加快了钙钛矿相的形成。不过,介孔\ce{TiO2}薄膜的形成一般需要450-500$^{\circ}$C的高温,这在一定程度上限制了这种方法的使用。Liu等人后续使用低温制备的ZnO薄膜替代了介孔\ce{TiO2}薄膜,使用“两步法”生长钙钛矿薄膜,也取得了不错的效果\cite{RN112}。Xiao等人报道了如图\ref{fig:1.2.3.2}h所示的“两步旋涂法”工艺,不同于上述在介孔\ce{TiO2}薄膜上使用“两步法制”备钙钛矿薄膜的工艺,在这里\ce{MAPbI3}钙钛矿薄膜是通过在\ce{PbI2}薄膜上旋涂含有\ce{MAI}的IPA溶液后退火得到的\cite{RN19}。在两步旋涂工艺中,\ce{MAI}和\ce{PbI2}的反应速度并不是很快,因此后续退火的步骤是十分必要的。另外一种“两步法”制备钙钛矿薄膜的工艺用到了\ce{MAI}蒸汽辅助(原理如图\ref{fig:1.2.3.2}i所示),Abbas等人和Chen等人,将使用热蒸发法或溶液法制备的\ce{PbI2}薄膜置于\ce{MAI}蒸汽氛围中,\ce{MAI}蒸汽会沉积在\ce{PbI2}薄膜上,进而形成\ce{MAPbI3}钙钛矿薄膜\cite{RN116,RN117}。

除了使用溶液法,还可以使用物理气相沉积法(热蒸发法)来制备钙钛矿。热蒸发法不受限于前驱体材料在溶剂中的溶解度,从原理上来说是一种适用性较广的直接沉积各种类型高质量钙钛矿薄膜的方法\cite{RN114}。在早期报道中,Liu等人使用了共同蒸发\ce{CH3NH3I}和\ce{PbCl2}前驱体的工艺来制备钙钛矿薄膜,相较于溶液法制备的薄膜,使用热蒸发工艺得到的薄膜更加均匀(图\ref{fig:1.2.3.2.2}),将其应用于太阳能电池也获得了更高的效率\cite{RN99}。热蒸发法还具有可以用于制备大面积器件的优势,可以最大程度上保证器件性能的均一性\cite{RN115}。

\section{光电探测器简介}

光电探测器是一种通过光电效应探测或测量光信号性质的传感器。下面将介绍一些表征光电探测器性能的重要参数,再在此基础上介绍几种常见的光电探测器,以区分这些器件在应用上的不同。

\subsection{光电探测器的性能参数}%光谱响应度$R$、比探测率$D^*$、EQE、响应时间、噪声电流密度等性能参数的定义及物理意义。

\textbf{(1)光电流密度和暗电流密度:}

光电流密度($J$)和暗电流密度($J_{\rm{d}}$)分别代表在有光照情况下和在无光照情况下流经光电探测器的电流密度,$J$与$J_{\rm{d}}$的差值被定义为光生电流密度($J_{\rm{ph}}$)\cite{RN73,RN74,RN75},它们的单位一般为\si{A\ cm^{-2}}。光电探测器的暗电流密度大小是衡量器件是否可以实际应用的关键参数,较高的暗电流密度会增加能源消耗并且会干扰信号的获取。

\textbf{(2)外量子效率和响应度:}

外量子效率(EQE)也被称为入射光子转化为电流的效率(IPCE)\cite{RN75},一般用\%衡量其大小。在实际应用中,更多地会用到响应度($R$),其表示为光信号转换为电信号的比例。在电信号为光生电流情况下,$R$可用以下公式来计算:

\begin{equation}\label{eq-1.3.1}
R = \frac{J_{\rm{ph}}}{P_{\rm{in}}}
\end{equation}

其中$P_{\rm{in}}$为入射光光强,$R$的单位一般为 \si{A\ W^{-1}}或\si{V\ W^{-1}}。EQE与$R$的关系为:

\begin{equation}\label{eq-1.3.2}
{\rm{EQE} } = R \frac{hc}{q\lambda}
\end{equation}

其中$h$为普朗克常数,$c$为真空中的光速,$q$为元电荷电荷量,$\lambda$为被探测光线的波长。一般研究中会测试光电探测器在不同探测波长下的$R$,并给出器件在入射波长为多少时$R$取得最大值。

%\textbf{噪声谱密度:}光电探测器不仅会产生光电流,也会产生很多类型可能会掩盖信号的电子噪声,其单位一般为,一般情况下减小器件的$J_D$可以显著降低噪声谱密度$i_n$,可使用频谱分析仪或者将测得的$J_D$随时间变化的数据进行快速傅里叶变换(FFT)来得到$i_n$,其单位一般为\si{A\ Hz^{-1/2}}。\cite{RN75}

\textbf{(3)比探测度:}%噪声等效功率(NEP)的数值表示光电探测器可以探测到入射光子的微弱程度,可以用$R$和$i_n$以下面形式表示:

%\begin{equation}\label{eq-1.3.4}
%{\rm{NEP} } = \frac{i_n}{R}
%\end{equation}

%NEP的单位一般为\si{W\ Hz^{-1/2}}。将NEP用测量带宽(B)(光电探测器可以探测到入射光子的波长或频率范围\cite{RN72) 归一化就可以得到$\rm{NEP_B}$:

%\begin{equation}\label{eq-1.3.5}
%{\rm{NEP_B} } = \frac{{\rm{NEP} }}{\sqrt{B}} = \frac{i_n}{R\sqrt{B}}
%\end{equation}
%$\rm{NEP_B}$的倒数即为光电探测器的
($D^*$)被用来衡量光电探测器检测到的入射光的微弱程度,单位为\si{cm\ Hz^{1/2}\ W^{-1}}或Jones,一般用以下公式来计算:

\begin{equation}\label{eq-1.3.3}
D^* = \frac{R\sqrt{A}}{i_{\rm{n}}}
\end{equation}

其中,A表示器件的有效面积,$i_{\rm{n}}$表示噪声电流。一般情况下减小器件的$J_{\rm{d}}$可以显著降低$i_{\rm{n}}$,可使用频谱分析仪或者将测得的$J_{\rm{d}}$随时间变化的数据进行快速傅里叶变换(FFT)来得到$i_{\rm{n}}$\cite{RN75}。

\textbf{(4)响应时间:}

通常研究者会使用上升时间($t_{\rm{raise}}$)和下降时间($t_{\rm{fall}}$)来衡量光电探测器对光信号响应的快慢\cite{RN73}。一般地,将$t_{\rm{raise}}$定义为从输出信号最大值的10\%上升到输出信号最大值的90\%所需要的时间,将$t_{\rm{fall}}$定义为从输出信号最大值的90\%下降到输出信号最大值的90\%所需要的时间\cite{RN118,RN73,RN75}。光电探测器的响应时间和器件中的电荷传输和收集有着密切关系。

\textbf{(5)线性动态范围:}

光电探测器对入射光线光强可以线性响应的范围被称为线性动态范围(LDR)\cite{RN118,RN73}。一般用对数函数来表示:

\begin{equation}\label{eq-1.3.8}
{\rm{LDR} } = 20\log\frac{J_{\rm{upper}} - J_D}{J_{\rm{lower}} - J_D}
\end{equation}

其中$J_{\rm{upper}}$为光电探测器响应偏离线性的电流值,$J_{\rm{lower}}$为光电探测器的分辨率极限\cite{RN118}。在实际应用中,我们期望光电探测器具有较大的LDR值,这样光电探测器就可以检测较强和较弱的光线。当然,如果器件的性能可以很好地表征出来,光电探测器也是可以在LDR之外工作。

\subsection{光电探测器的分类}

\begin{figure}[ht]
\centering
  \includegraphics[width=\textwidth]{Fig_1.3.1.jpg}
  \caption{\rm \ (a)光电导、(b) p-i-n型光电二极管和(c) n沟道光电晶体管的能级示意图和它们相应的(d)-(f) \emph{J-V}特性曲线、(g)-(i)响应度和(j)-(l)随光强的变化\cite{RN75}。}
  \label{fig:1.3.1}
\end{figure}

如图\ref{fig:1.3.1}所示,按照器件结构及工作原理可将光电探测器分类为光电导、光电二极管和光电晶体管三种类型\cite{RN75}。图\ref{fig:1.3.1}d展示了光电导器件的\emph{J-V}特性曲线。使用欧姆定律和电导率($\sigma$)的定义,可以得到电流密度的表达式:

\begin{equation}\label{eq-1.3.6}
J = \sigma\varepsilon = \sigma\frac{V}{d} = (\sigma_{\rm{d}}+\sigma_{\rm{ph}})\frac{V}{d}
\end{equation}

其中$\varepsilon$是电场强度,$V$是电压,$d$为活性层的厚度或通道宽度,$\sigma_{\rm{d}}$和$\sigma_{\rm{ph}}$分别代表材料在暗态和光照下的电导率。对于光电导器件,光照可以显著提高材料的载流子密度,进而提高材料的电导率。在一定偏压下,通过载流子再循环过程可产生光电导增益($G$),其可以用载流子的寿命$τ$和渡越时间$t_{\rm{tr}}$的比值来定义。而载流子的渡越时间又和载流子迁移率$\mu$以及沟道长度或活性层的厚度密切相关,故可将$G$表示为:

\begin{equation}\label{eq-1.3.7}
G =\frac{\tau}{t_{\rm{tr}}} =\frac{\tau\mu{V}}{d^2}
\end{equation}

载流子寿命和迁移率都是材料本身的性质,为提高器件的$G$,可以考虑减小沟道长度。然而,降低$d$会使器件的暗电流增加,进而影响信号的读取。这种窘境可用光电二极管来解决。图\ref{fig:1.3.1}b所示为一种典型的p-i-n结构的光电二极管,本征半导体夹在p型半导体和n型半导体之间,本征半导体一般较薄,在暗态下其中的载流子会被内建电场耗尽形成一个高阻区。一般情况下,光电二极管在负偏压下工作,这时节中的载流子被完全耗尽,器件的暗电流密度非常低(图\ref{fig:1.3.1}e)。对于光电晶体管,半导体中的电荷传输可用栅压来进行调控,使得$G$可被操控。一般光电晶体管的结构都是基于场效应晶体管(FET),FET一般由介电层、薄膜通道和三个电极(源极、漏极和栅极)构成。其中,介电层一般为绝缘物质,薄膜通道一般由对光敏感的半导体材料构成。此外,如果光电晶体管具有额外的光吸收层,则又可被称为敏化光电晶体管。图\ref{fig:1.3.1}c所示为一种n型通道的敏化光电晶体管,图\ref{fig:1.3.1}f展示了它的转移特性曲线,由于栅压的存在,使其暗电流得以抑制。敏化层和通道之间的结会产生垂直光掺杂效应,注入的电子在通道中输送和再循环进而产生增益。 下面将对这三种类型的光电探测器在响应度,线性度和相应时间上进行比较。如图\ref{fig:1.3.1}g所示,对于光电导器件来说,其响应度随着入射光强的增加而减小,这是由于入射光强增加导致载流子的复合和散射增加造成的。相较于光电导器件,光电二极管器件的响应度一般较小,但是其响应度随着入射光强的增加在很大范围内都是线性增加的。而当在入射光强度较高时,电极的电阻阻碍了电荷的抽取,进而使得器件的响应度显著减小(图\ref{fig:1.3.1}h)。图\ref{fig:1.3.1}i展示了光电晶体管的响应度随入射光光强的变化,可以看出其变化趋势和光电导器件的变化趋势相同,但是光电晶体管器件具有较好的电荷抽取和可调的增益。图\ref{fig:1.3.1}j-图\ref{fig:1.3.1}l展示了三种器件的光电流随入射光强变化的双对数坐标曲线。对于光电二极管器件,曲线的斜率为1,表明其响应度在一定范围内为定值。对于光电导和光电晶体管器件来说,曲线的斜率<1, 说明器件的响应度随着光强的增加而减小。另外,由于光电导增益的存在,图\ref{fig:1.3.1}j和图\ref{fig:1.3.1}l所示的电流密度的数值要大于图\ref{fig:1.3.1}k所示的电流密度数值。此外,值得一提的是,光电二极管型器件的响应速度一般比光电导型和光电晶体管型器件的响应速度要快。光电导型器件和光电晶体管型器件的响应速度可以通过减少沟道宽度和增加偏压来加快。而由于光电晶体管型器件的多数特性都可使用栅压调节,因此较光电导器件来说,在较高场强下仍具有较低的暗电流。

\section{钙钛矿光电探测器的发展简介}

根据钙钛矿光电探测器的探测光谱范围可将其分为:可见光电探测器、紫外光电探测器和红外光电探测器。另外,一些特殊波长的光线(比如,X-射线和$\gamma$-射线)也可被探测到。下面将根据钙钛矿光电探测器的探测光谱的范围来介绍钙钛矿光电探测器的发展。

\subsection{可见光电探测器}

最早在2014年,Hu等人就在柔性氧化铟锡(ITO)导电衬底上沉积了\ce{MAPbI3}钙钛矿薄膜,制作了探测范围从紫外到可见光范围的光电导类型光电探测器\cite{RN122}。这种光电探测器可探测从310 nm到780 nm波长的光线,在入射光强为0.01\ \si{mW \ cm^{-2}}时,其$R$最高可达7 \si{A\ W^{-1}},并且响应时间<0.2 s。此外,该器件还展现了良好的柔韧性,在弯折120次后,器件性能基本保持不变。以上实验结果表明,该器件结构简单、响应度高、响应速度快且稳定性好,说明钙钛矿薄膜具有应用于高性能光电探测器的巨大潜力。在同一年,Dou等人使用溶液法制备了\ce{MAPbI_{3-x}Cl_x}钙钛矿薄膜,并用于制作光电二极管型光电探测器,设计了器件界面并优化了钙钛矿薄膜的厚度,使得所制作的器件可探测300 nm到800 nm波长的光线,具有良好的器件性能(图\ref{fig:1.4.1}a和b)\cite{RN129}。他们还使用了聚合物材料作为空穴阻挡层,使得器件暗电流密度显著降低,并使器件在没有施加偏压的情况下就可以达到\num{e14}Jones的$D^*$(在550 nm波长下)。此外,得益于光电二极管这种垂直器件结构,器件的$t_{\rm{raise}}$和$t_{\rm{fall}}$分别达到了180 ns和160 ns。Mohammed课题组制作了基于\ce{MAPbI3}/\ce{MAPbBr3}异质结的三明治结构的光电探测器。\cite{RN138}这种光电探测器没有电子传输层(ETL)和空穴传输层(HTL),但器件性能却与具有ETL和HTL的器件相当,证明了单独钙钛矿层也可组成器件并应用于光电探测器领域。Wu等人制作了基于\ce{MAPbI3}和
\ce{MAPbI_{3-x}Cl_x}薄膜的光电晶体管(图\ref{fig:1.4.1}c和d)\cite{RN139}。有趣的是,在这种光电晶体管器件中观察到了双极性载流子特性,这意味着该器件既可以在p型模式下又可以在n型模式下工作。当栅压为-40 V时,这种光电晶体管的$R$可达320 \si{A\ W^{-1}},响应时间小于10 μs。

上述光电探测器都是基于钙钛矿薄膜的,当器件中的活性层变为钙钛矿的其他纳米结构,比如,纳米晶体、纳米线以及纳米棒等时,其光电性能也会发生改变,进而使得器件性能也发生改变。Horváth等人首次制备了\ce{MAPbI3}纳米线,并制作了相应的光电探测器\cite{RN140}。器件$R$在微弱入射光(70 nW到70\ μW)下达到了5 \si{A\ W^{-1}},响应时间低于500 μs,比基于\ce{MAPbI3}薄膜的器件响应速度要快。和有机无机杂化钙钛矿材料类似,一维全无机钙钛矿材料也可应用于光电探测器。Tang等人,制备了\ce{CsPb(Br/I)_3}纳米棒并将其应用于光电探测器,器件的开关比高达2000\cite{RN141}。考虑到钙钛矿纳米线具有较好的柔韧性,Zhu等人以及Hu等人制备了基于钙钛矿纳米线的光电探测器\cite{RN142,RN143}。随着一维钙钛矿纳米材料在光电探测器中的应用,基于钙钛矿纳米晶的光电探测器也逐渐发展起来。Jang等人制备了\ce{MAPbX3}钙钛矿纳米晶,并使用该钙钛矿纳米晶制备了用于光电探测器的高质量薄膜\cite{RN61}。在这以后,基于\ce{CsPbX3}全无机钙钛矿量子点的光电探测器也被制作了出来\cite{RN144}。

\begin{figure}[ht]
\centering
  \includegraphics[width=\textwidth]{Fig_1.4.1.jpg}
  \caption{\rm \ 基于\ce{MAPbI_{3-x}Cl_x}钙钛矿薄膜的光电二极管型光电探测器的(a)器件结构示意图和(b)不同波长下的器件的EQE和$D^*$\cite{RN129},光电晶体管型光电探测器的(c)器件结构示意图和(d)其双极性转移特性曲线\cite{RN139}, 基于二维\ce{CsPbBr3}全无机钙钛矿薄膜的柔性光电探测器的(e)器件结构示意图和(f)其\emph{I-V}特性曲线\cite{RN146}。}
  \label{fig:1.4.1}
\end{figure}

除了基于一维钙钛矿材料或钙钛矿纳米晶的光电探测器,二维钙钛矿由于具有较好的载流子迁移率、优异的导电性能以及很强的量子限域效应也被广泛应用于钙钛矿光电探测器。此外,二维钙钛矿材料还具有应用于柔性器件的优势,因为二维钙钛矿材料具有易于与柔性衬底相匹配的光滑表面以及相较于对应块体材料较强的光响应。Liu等人将二维\ce{MAPbX3}钙钛矿作为活性层应用于光电导型光电探测器,器件在较低偏压(1 V)以及波长为405 nm微弱光线(10\ \si{pw\ cm^{-2}})下$R$可达22 \si{A\ W^{-1}}\cite{RN145}。Song等人制备了基于二维\ce{CsPbBr3}全无机钙钛矿薄膜的高性能柔性光电探测器,如图\ref{fig:1.4.1}e和f\cite{RN146}。该器件的开关比超过\num{e3}且具有优良的柔韧性。

对于基于钙钛矿的光电探测器,材料的晶界和缺陷都会阻碍载流子的传输从而影响器件的性能。考虑到这个因素,没有晶界和缺陷较少的钙钛矿单晶材料是一种应用于光电探测器的理想材料。Lian等人和Zhang等人制作了基于\ce{MAPbI3}和\ce{MAPb(Br_xI_{1-x})_3}钙钛矿单晶的光电探测器,都采用了光电导的器件结构并具有较好的光电性能\cite{RN147,RN148}。与\ce{MAPbX3}相比,\ce{FAPbX3}钙钛矿材料具有更好的稳定性和更宽的光吸收范围,Yang等人制备了尺寸为5 mm的\ce{FAPbI3}钙钛矿单晶,并将其应用到了光电探测器上\cite{RN150}。后续发展中,Liu等人制备了具有较大体积的钙钛矿单晶,并将其切为薄晶片后应用于光电探测器,也取得了不错的性能\cite{RN151}。

\begin{figure}[htb]
\centering
  \includegraphics[width=\textwidth]{Fig_1.4.2-1.jpg}
  \caption{\rm \ (a)基于薄膜光电二极管类型的窄带光电探测器原理图\cite{RN152}和(b)基于薄膜的窄带光电二极管EQE图谱和相应材料吸收光谱\cite{RN153},(c)基于厚膜光电二极管类型的窄带光电探测器原理图\cite{RN152}和(d)基于单晶厚膜的窄带光电二极管的EQE图谱\cite{RN130}。}
  \label{fig:1.4.2}
\end{figure}

除了对较宽波长范围内光线都具有响应的宽频光电探测器,对特定波长响应的窄带光电探测器也随着对钙钛矿材料和器件的研究逐渐发展起来。窄带光电探测器需要分辨不同颜色的光线,现有的成熟光电系统一般都需要使用不同颜色滤光片来实现对不同颜色光线的探测。这在一定程度上增加了制作光电探测器的复杂程度,并且阻碍了探测器像素密度的提升。

为了解决这个问题,Lin等人\cite{RN153}和Fang等人\cite{RN130}独立报道了不需要滤光片的窄带钙钛矿光电探测器,图\ref{fig:1.4.2}a与\ref{fig:1.4.2}c分别展示了这两种器件的工作原理图。为实现窄带光电探测,需要剔除短波长光线对光电流的贡献,为此他们都采用了在有机光电二极管中使用charge collection narrowing(CCN)的方案\cite{RN154}。电子-空穴对在钙钛矿材料体内均匀产生并被有效收集从而产生光电流,在半导体材料表面产生的光生载流子因不能被有效收集而不贡献光电流,这种不充分的电荷收集现象使得光电探测器的光谱响应范围变窄。通过调控半导体材料的吸收系数和厚度可以实现对窄带光电探测器响应光谱范围的调节。Lin等人将小分子染料与钙钛矿混合以提高材料的光吸收系数,实现了对光电探测器响应光谱范围(响应光谱半高宽约为100 nm)的调控(图\ref{fig:1.4.2}b)。图\ref{fig:1.4.2}d展示了Fang等人制作的钙钛矿单晶厚膜窄带光电二极管的EQE图谱,表明了这种窄带光电探测器可实现在430 nm到630 nm波长范围内半高宽<20 nm的光谱响应。对于厚膜器件来说,只有能量靠近材料带隙的光子才能在整个器件中被均匀吸收并产生光电流,能量小于带隙的光子可穿过器件不产生光电流,而能量高于带隙的光子根据Beer-Lambert定律产生指数分布的载流子,几乎不贡献光电流,从而实现了如图\ref{fig:1.4.2}d所示的窄带光谱探测。

此外,在其他结构的光电探测器中也实现了窄带探测,Wu等人制作了基于钙钛矿厚膜(>10\ $\si{\um}$)的光电导窄带光电探测器,由于载流子扩散长度的限制使被材料表面吸收的光子几乎不贡献光电流,实现了半高宽为13.6 nm超窄带光谱响应\cite{RN155}。

\subsection{紫外光电探测器}

紫外光一般是指波长范围在10 nm到400 nm之间的电磁辐射,按照不同标准可将其划分为如图\ref{fig:1.4.3.0}所示的不同区域\cite{RN175}。考虑到氯化物钙钛矿带隙大都在3 eV附近,所以一般用于紫外光检测的光电探测器都是基于\ce{MAPbCl3}、\ce{CsPbCl3}等宽带隙钙钛矿材料的。

\begin{figure}[ht]
\centering
  \includegraphics[width=\textwidth]{Fig_1.4.3-0.jpg}
  \caption{\rm 按照不同标准将紫外光区域分类\cite{RN175}。}
  \label{fig:1.4.3.0}
\end{figure}

早在2015年,Maculan等人就制作了基于\ce{MAPbCl3}钙钛矿单晶的第一个紫外光电探测器,证明了宽带隙钙钛矿材料可被应用于紫外光电探测(图\ref{fig:1.4.3}a和b)\cite{RN127}。后续Sargent课题组制作了基于\ce{MAPbCl3}钙钛矿薄膜的紫外光电探测器,实现了从350 nm到430 nm波长范围光线探测,并且其$R$在弱光下达到了18\ \si{A\ W^{-1}}\cite{RN126}。Zhang等人还将两步旋涂法制备的\ce{MAPbCl3}多晶薄膜应用于制作光电二极管型紫外光电探测器,所得器件对300 - 400 nm波长的光线具有很好的光响应(图\ref{fig:1.4.3}c和d)\cite{RN156}。Chen等人使用了“限域”法制备了高质量\ce{MAPbCl3}单晶薄片,并将其应用到光电二极管类型的紫外光电探测器中(图\ref{fig:1.4.3}e和f)\cite{RN157}。这种基于单晶薄片的光电二极管器件具有较低的噪声(6.5\ \si{fA\ Hz^{-1/2}})、较高的比探测度(\num{6e12}\ Jones)以及很快的响应速度(15 ns)。相较于有机无机杂化钙钛矿,全无机钙钛矿具有更好的稳定性。Gui等人使用类似的方法制备了\ce{CsPbCl3}全无机钙钛矿单晶薄片,并将这种薄膜用于光电导型紫外光电探测器,所得器件展现出了很好的稳定性,在空气中200℃温度下退火30分钟后,器件光电流未出现明显变化\cite{RN89}。由于CsCl在常见有机溶剂中的溶解度较低,所以较难使用溶液法制备\ce{CsPbCl3}全无机钙钛矿薄膜\cite{RN137,RN89,RN90}。而热蒸发法这种制备薄膜的工艺则不受材料溶解度限制,Yang等人就使用热蒸发法制备了\ce{CsPbCl3}全无机钙钛矿薄膜,并将其应用于光电二极管型紫外光电探测器,拓宽了\ce{CsPbCl3}全无机钙钛矿材料在光电器件中的应用\cite{RN90}。此外,将\ce{CsPbCl3}钙钛矿量子点\cite{RN159}或二维\ce{CsPbCl3}钙钛矿薄片\cite{RN160}应用到紫外光电探测器也是也是解决材料溶解度较低的方案。除了常见的氯化物宽带隙钙钛矿材料,诸如\ce{(CH3NH3)2MnCl4}\cite{RN161}、\ce{Cs3Cu2I5}\cite{RN162}等非铅宽带隙钙钛矿也可应用于制作紫外光电探测器。

\begin{figure}[ht]
\centering
  \includegraphics[width=\textwidth]{Fig_1.4.3.jpg}
  \caption{\rm (a)基于\ce{CH3NH3PbCl3}钙钛矿单晶的紫外光电探测器器件结构示意图和(b)其在光照(波长365 nm)和暗态下的\emph{J-V}曲线\cite{RN127},(c)基于\ce{CH3NH3PbCl3}钙钛矿多晶薄膜的光电二极管型器件结构示意图和(d)其在不同偏压下的EQE图谱\cite{RN156},(e)基于“限域”法制备的\ce{CH3NH3PbCl3}单晶薄片的光电二极管类型紫外光电探测器结构示意图和(f)其在0偏压下的瞬态光电流图谱\cite{RN157}。}
  \label{fig:1.4.3}
\end{figure}

\subsection{红外光电探测器}

红外光电探测器在红外成像、生物检测以及信息通讯等方面的具有巨大应用前景,因此发展基于钙钛矿的红外光电探测器也十分重要。将钙钛矿材料应用到红外光电探测器最简单的途径就是展宽钙钛矿材料的光谱吸收范围,Xu等人将常规\ce{MAPbI3}替换为\ce{MA_{0.5}FA_{0.5}Pb_{0.5}Sn_{0.5}I_3},使得材料的光谱吸收范围延伸到1000 nm\cite{RN163}。此外,将钙钛矿材料与其他材料复合也是一种拓宽光谱吸收范围的途径。Bessonov等人使用PbS量子点与\ce{MAPbI3}复合,将器件的光谱探测范围拓展到了1500 nm。\cite{RN164}使用PDPP3T\cite{RN165}和PDPP3T:PC71BM\cite{RN166}与\ce{MAPbI3}构筑双活性层结构的光电探测器,可以将器件的光谱响应范围分别延展到940 nm和950 nm。此外,也可使用NaYF4:Yb/Er纳米颗粒\cite{RN167}和铒镱硅酸盐纳米片\cite{RN168}等上转换材料与钙钛矿混合作为光敏层,器件的光谱检测范围可分别展宽至980 nm和1540 nm。

\subsection{特殊波长光电探测器}

钙钛矿材料具有较高的载流子迁移率、较长的载流子寿命以及成分原子序数较高等特性,因此其也成为制备检测X-射线和$\gamma$-射线等特殊波长光电探测器的理想材料。早在2013年,Stoumpos等人便提出可将\ce{CsPbBr3}单晶应用于高能辐射检测\cite{RN169}。2015年,Yakunin等人报道了基于溶液法制备钙钛矿材料的X-射线探测器,证明了无论是光电二极管结构还是光电导结构的器件都可用于X-射线检测,并展现出高达25 μ\ce{C\ mGy_{air}^{-1}\ cm^{-3}}的灵敏度\cite{RN170}。Wei等人使用带隙较宽的\ce{MAPbBr3}单晶应用到X-射线探测器,得益于单晶材料优异的光电特性,器件探测极限可低至0.5 μ\si{Gy_{air}\ s^{-1}},且灵敏度高达80 μ\ce{C\ mGy_{air}^{-1}\ cm^{-2}}\cite{RN171}。此外,Pan等人还制作了基于\ce{Cs2AgBiBr6}钙钛矿单晶的X-射线探测器,在这种钙钛矿材料中离子移动可被抑制,施加较高偏压促进电荷抽取的同时并不会使噪声电流显著增加,因此这种基于\ce{Cs2AgBiBr6}钙钛矿单晶的X-射线探测器表现出极低的检测极限(59.7 \si{nGy_{air}\ s^{-1}})\cite{RN172}。在更进一步的发展中,Nazarenko等人\cite{RN174}以及Wei等人\cite{RN173}分别将\ce{FACsPbX3}和\ce{MAPbBr_{3−x}Cl_x}单晶应用于$\gamma$-射线检测,也获得了不错的性能。

\section{紫外光电探测器的发展及研究挑战}%简单介绍紫外光电探测器的应用及发展。紫外光电探测器制备材料及工艺问题。

硅半导体具有成熟的制备工艺,因此传统的紫外光电探测器一般都是基于硅材料的器件。但是硅材料通常具有较窄的带隙(1.12 eV),因此硅基光电二极管紫外探测器常需要配合滤光片使用\cite{RN179},这增加了器件结构的复杂程度和制造成本。此外使用滤光片还会减弱入射光线的强度,给器件的响应度带来不可忽略的影响。诸如GaN、SiC以及ZnO等宽带隙直接带隙半导体几乎不吸收可见光,将其应用于紫外光电探测器免去使用滤光片所带来的问题。但GaN和SiC常使用金属-有机化学气相沉积和分子束外延等需要较高温度和真空度的工艺来制备,使得材料的生产成本较高\cite{RN131,RN25}。而ZnO则可使用简单的溶液旋涂法来制备,使其成为理想的用于制作紫外光电探测器的氧化物半导体材料。\iffalse 但是p型ZnO稳定性较差,使基于ZnO同质结的器件较难制作出来。\fi 无机半导体具有较好的稳定性,另一方面,有机半导体可使用简单的方法来制备,并可作为受主材料。因此,将ZnO和有机半导体复合使用,使器件兼具无机和有机半导体的优点,可以拓展ZnO在紫外光电探测器上的应用。但是,一般情况下有机半导体的载流子迁移率都比较低,使得基于ZnO/有机半导体的异质结器件响应度较低。针对这个问题,可通过提高有机半导体的结晶程度以及向有机半导体中掺杂其他载流子迁移率较高的材料来解决\cite{RN183}。

钙钛矿材料具有优异的光电特性,可用于制作高性能的光电器件。在上一节中,介绍了基于宽带隙钙钛矿材料的紫外光电探测器的发展概况,可见氯基宽带隙钙钛矿因具有合适的带隙宽度而被广泛应用于紫外光电探测器中。研究人员制备了单晶、多晶薄膜、单晶薄片等不同形态的\ce{MAPbCl3}材料,并基于这种有机无机杂化钙钛矿制作了光电二极管、光电导等多种类型的紫外光电探测器。相较于有机无机杂化钙钛矿,全无机钙钛矿材料具有更好的稳定性,但是由于CsCl在常见有机溶剂中溶解度较低,使用旋涂法较难制备\ce{CsPbCl3}全无机钙钛矿薄膜。将二维或量子点钙钛矿材料和使用“限域”法制备的全无机钙钛矿薄片应用于紫外光电探测器,是一种解决材料溶解度较低限制的方法。当然,为解决此类问题,也有人员尝试了不受材料溶解度限制的热蒸发法来制\ce{CsPbCl3}全无机钙钛矿薄膜,这进一步拓宽了全无机钙钛矿材料在光电器件中的应用。但是使用热蒸发法来制备宽带隙全无机钙钛矿材料的研究相对较少, 在材料的制备工艺、器件结构的选择与优化等方面还需要进一步探索。

\section{本论文的研究内容及意义}

针对用于紫外光电探测器的有机半导体材料载流子迁移率较低的问题,尝试了在有机半导体中掺杂\ce{MAPbCl3}钙钛矿量子点。钙钛矿量子点的引入使得组成器件的各层之间能级更加匹配,并且提高了有机半导体的载流子迁移率。实验结果表明,向有机半导体中掺杂适量钙钛矿量子点可在不影响器件暗电流密度的情况下提高器件的响应度、比探测度以及响应速度等性能参数。进一步地,还将类似器件结构应用于柔性器件,在多次弯折后,器件光电性能衰减较少,证明了制作柔性器件可行性。考虑到有机无机杂化钙钛矿材料存在着稳定性较差的问题,还测试了器件的长期稳定性,器件在放置2个月后其光电流依旧为初始光电流的82\%,显示出了较好的稳定性。此项研究为提高基于ZnO/有机半导体复合层的紫外光电探测器性能提供了一种可行的解决方案,并且推动了宽带隙钙钛矿材料在紫外光电探测方面的应用研究。

为进一步探索热蒸发制备宽带隙全无机钙钛矿薄膜的工艺和研究全无机钙钛矿材料在紫外光电探测器上的应用,使用热蒸发法沉积了\ce{CsPbCl3}全无机钙钛矿薄膜,并将其应用于紫外光电探测器。首先探究了钙钛矿前驱体沉积速率及比例对成膜质量的影响,在得到最佳参数后,将沉积得到的薄膜应用于紫外光电探测,并研究了将钙钛矿薄膜进行不同温度退火处理对器件性能的影响。通过优化器件结构和退火温度,将器件的EQE从不到5\%提高到了20\%左右。进一步的研究还表明,在较高退火温度下,器件性能会出现下降,根据对沉积于ZnO上的钙钛矿薄膜的研究,推测出了一种可能导致器件性能下降的原因。这项工作拓宽了全无机钙钛矿材料在紫外光电探测器中的应用,为宽带隙全无机钙钛矿材料在光电器件中的应用奠定了一定实验研究基础。

%%%=== 基于钙钛矿量子点掺杂的紫外光电探测器的研究 ========%%%

\chapter{钙钛矿量子点掺杂的紫外光电探测器的研究}

\section{引言}

一般商业化紫外光电探测器是基于GaN、SiC等宽禁带无机半导体的,但是,制作基于无机半导体的器件的工艺一般比较复杂而且昂贵。比如,制备无机半导体一般会需要较高的真空度和较高的沉积温度,增加了能源消耗\cite{RN131,RN25}。为降低生产制造成本,使用溶液法加工用于紫外光电探测器的有机半或者有机无机杂化半导体成为一种可行的替代方案\cite{RN37,RN38,RN39} 。例如,ZnO由于具有3.37 eV的直接带隙、较好的化学稳定性以及可使用溶液法以及真空法等低成本工艺制备的特点,而成为了最有希望用于制作的紫外光电探测器的宽带隙无机氧化物\cite{RN8,RN7}。然而,由于p型ZnO比较不稳定,基于同质结的光电探测器很少有人报道\cite{RN25}。

针对p型ZnO稳定性较差的问题,可通过将p型有机半导体和n型ZnO结合使用来解决。Dhar等人报道了基于ZnO纳米棒和聚3,4-乙烯二氧噻吩/聚苯乙烯磺酸盐(PEDOT:PSS)复合材料的肖特基结二极管\cite{RN28}。这种光电探测器在-1 V的偏压和340 nm光照下表现出了很高的响应度({36} \si{A\ W^{-1}}) 和比探测度(\num{1.3e12} Jones)。最近,Zheng等人报道了基于PTAA/ZnO异质结的窄带紫外光电探测器\cite{RN25}。然而,由于PTAA层较差的电荷输运性能,使得这种类型的光电探测器的响应度较低。针对有机半导体载流子迁移率较低的问题,可通过提高有机半导体的结晶度或向有机半导体中掺入其他载流子迁移率较高的材料等方式来解决\cite{RN183}。例如,Intaniwet等人将PTAA与载流子迁移率较高的6,13-双(三异丙基硅烷基乙炔基)并五苯(TIPS-pentacene)混合并用于制作X-射线探测器\cite{RN34}。所制作的X-射线探测器具有更好的性能,这和复合材料较高的载流子迁移率有关。

值得一提的是,由于具有较低的加工成本、柔韧性较好和化学种类丰富的特点,有机半导体材料和器件的相关研究已经取得了引人瞩目的进展\cite{RN73,RN41,RN44}。 Armin等人制造了具有较低暗电流的大面积柔性有机光电探测器\cite{RN29}。 Xiong等人制作了具有高灵敏度、低噪声电流的有机光电探测器\cite{RN35} 。Wu等人报道了一种具有光电倍增的有机图像传感器\cite{RN30}。 Li等人研究出了一种探测范围覆盖紫外-可见-红外范围的宽带有机光电二极管\cite{RN1} 。然而,基于有机半导体的高性能紫外光电探测器却很少人报道。此外,Wu\cite{RN31}等人和Ray\cite{RN32}等人报道了具有高响应度的器件,但是这些器件都需要使用较高的偏压来促进电荷的输运,并且都还存在着暗电流密度较高的问题。

在过去的几年里,由于金属卤化物钙钛矿纳米晶体材料在激光、太阳能电池、发光二极管和光电探测器等高性能光电器件中的广泛应用,从而吸引了众多研究人员的兴趣。钙钛矿材料具有较长的电子-空穴扩散长度、较长的载流子寿命、较高的光吸收系数和较高的载流子迁移率等优异的光电特性,为提高器件性能提供了极大的空间\cite{RN21,RN22,RN40}。在这项工作中,我尝试了将\ce{MAPbCl3}钙钛矿量子点掺杂到了PTAA厚膜中,期望提高有机半导体薄膜的电荷输运性能和器件性能。不同于体异质结器件,使用ZnO/PTAA双层结构的器件还具有较低的暗电流和漏电流,并且在较低的偏压下钙钛矿量子点就可以有效提高电荷的输运性能。因此可以预见,这种掺杂策略可以使有机无机杂化型紫外光电探测器具有很低的暗电流以及较好的探测性能。\vspace{-2mm}

\section{实验}

\subsection{材料}

聚[双(4-苯基)(2,4,6-三甲基苯基)胺](PTAA,M$_n$ = 15,000 \~{} 25,000)、氯化铅(\ce{PbCl2},>99.99\%)和浴铜灵(BCP,>99\%)购买于西安宝莱特光电科技有限公司。二氧化锡胶体分散液(\ce{SnO2},15\% in \ce{H2O} colloidal dispersion )、甲胺氯(\ce{CH3NH3Cl},99\%)和二甲基亚砜(DMSO)购买于阿法埃萨(中国)化学有限公司。醋酸锌(\ce{Zn(CH3COO)2},99\%)购买于九鼎化学。乙酸镍(99.9\%)、乙醇胺(\ce{NH2CH2CH2OH},99\%)和油胺(OAm,80-90\%)购买于阿拉丁试剂。\iffalse 醋酸镍四水合物(\ce{C4H14NiO8},AR,99\%)和\fi 油酸(OA,90\%)购买于天津希恩思奥普德科技有限公司。甲苯(AR)、丙酮(AR)、乙醇(AR)和盐酸(HCl,质量分数36\% \~{} 38\%)购买于国药集团化学试剂有限公司。乙二醇单甲醚(\ce{CH3OCH2CH2OH},99\%)购买于泰坦。氧化钼(\ce{MoO_{\rm{x}}},99.99\%)购买于上海易势化工。碳60(C60,99.5\%)购买于南京先丰纳米材料科技有限公司。

\subsection{器件制作}

所有器件都是在商业定制图案化的氧化铟锡(ITO)玻璃衬底或聚对苯二甲酸乙二醇酯(PET)柔性导电衬底上制作的。使用稀释后的盐酸在PET柔性导电衬底上刻蚀出所需的图案。先使用洗洁精清洗ITO玻璃衬底或PET柔性衬底,然后使用丙酮和乙醇分别超声清洗10分钟。清洗干净的衬底使用压缩空气吹干并放置在臭氧清洗机中处理10分钟后备用,然后按照ITO/ZnO/掺杂不同量\ce{MAPbCl3}钙钛矿量子点的PTAA/\ce{MoO_{\rm{x}}}/Ag或ITO/\ce{SnO2}/掺杂不同量\ce{MAPbCl3}钙钛矿量子点的PTAA/\ce{MoO_{\rm{x}}}/Ag或ITO/\ce{NiO_{\rm{x}}}/掺杂不同量\ce{MAPbCl3}钙钛矿量子点的PTAA/LiF/C60/BCP/Cu的器件结构在衬底上依次沉积薄膜。ZnO前驱体溶液按照之前报道的方法来配置\cite{RN1}。将167.1 mg醋酸锌粉末溶解在2 mL乙二醇单甲醚和56 $\si{\uL}$乙醇胺混合溶液中,并将上述溶液放置在空气中使用磁子以800 rpm搅拌12小时,再使用0.22 $\si{\um}$的聚四氟乙烯过滤头过滤后备用。以转速为2000 rpm、旋涂时长为15 \~{} 60 s的旋涂程序旋涂ZnO前驱体溶液来制备薄膜,再将旋涂得到的薄膜放在热台上顺序进行80$^{\circ}$C(10分钟)、140$^{\circ}$C(10分钟)和200$^{\circ}$C(20分钟)的退火后就可以得到ZnO薄膜。\ce{SnO2}薄膜使用Jiang等人报道的方法来制备\cite{RN111}。将\ce{SnO2}胶体分散液稀释至2.67\%,以转速为3000 rpm、旋涂时长为30 s的旋涂程序旋涂稀释\ce{SnO2}胶体分散液的来制备薄膜,再将旋涂得到的薄膜放在150$^{\circ}$C热台上退火30分钟后就得到了\ce{SnO2}薄膜。\ce{NiO_{\rm{x}}}薄膜同样是采用溶液法来制备,将99.5 mg乙酸镍粉末溶解在2 mL乙二醇单甲醚和24 $\si{\uL}$乙醇胺混合溶液中,并将上述溶液放置在70$^{\circ}$C的热台上加热2小时后使用0.22 $\si{\um}$的聚四氟乙烯过滤头过滤后备用。用转速为3000 rpm、旋涂时长为30 s的旋涂程序旋涂乙酸镍溶液来制备薄膜,将旋涂得到的薄膜放在150$^{\circ}$C热台上退火5分钟后再转移到300$^{\circ}$C的马弗炉中退火1小时后就得到了\ce{NiO_{\rm{x}}}薄膜。再将这些沉积了ZnO 薄膜或\ce{SnO2}薄膜或\ce{NiO_{\rm{x}}}薄膜的衬底转移到手套箱(水和氧含量都小于0.1 ppm)中沉积其他薄膜层和电极。钙钛矿量子点也采用Zhang等人之前报道的“过饱和溶液重结晶”的方法来制备\cite{RN10}。 一般的,0.32 mmol\ \ce{PbCl2}和0.32 mmol\ \ce{MACl}溶解在4 mL的DMSO中,然后将400 $\si{\uL}$ OA和40 $\si{\uL}$ OAm加入到上述溶剂中。用移液枪取0.5 mL上述前驱体溶液快速注入到搅拌中的5 mL甲苯中,就得到了钙钛矿纳米颗粒分散液,然后使用0.22 $\si{\um}$的聚四氟乙烯过滤头过滤钙钛矿纳米颗粒分散液以备用。使用旋涂法制备PTAA薄膜,先将上述钙钛矿量子点分散液稀释到不同的浓度,然后加入到PTAA粉末中配置成浓度为20 \si{mg\ mL^{-1}}的溶液,以150 rpm搅拌2小时以备用,再以1000 rpm、旋涂时长为30 s的旋涂程序来旋涂含有和不含有钙钛矿量子点的PTAA溶液来制备薄膜,将薄膜放置在100\textcelsius 的热台上退火10分钟后就得到了掺杂和未掺杂钙钛矿量子点的PTAA薄膜。最后使用热蒸发的方法将\ce{MoO_{\rm{x}}}(7.5 nm)/Ag(60 nm)或\ce{LiF}(1.2 nm)/C60(30 nm)/BCP(3.2 nm)/Cu(60 nm)顺序沉积在PTAA层上,最终制作的器件的有效面积为0.1 或0.08 \si{cm^2}。

\subsection{测试与表征}

材料的吸收光谱使用FR-Basic-UV/NIR-HR spectroscope (ThetaMetrisis)测试。钙钛矿\ce{MAPbCl3}量子点高分辨率透射电子显微镜照片(HRTEM)使用JEM2010--FEF高分辨场发射电子显微镜拍摄。材料的结构使用X射线衍射(XRD)来表征,所使用的仪器为D8(Cu K$\upalpha$,$\uplambda$ = 1.5418 \ {\AA})X射线仪,扫描范围从10{\degree}到55{\degree}。薄膜的厚度使用台阶仪(Alpha-step D-600 profile meter,KLA-Tencor)或椭偏仪(J. A. Woollam Co., Inc.)来获得。\emph{J-V}特性曲线(包括暗态下和光照下的)使用半导体分析仪(B1500A)在室温下测试获取。在测试有光照下的\emph{J-V}特性曲线时,使用氙灯(Solar-500,北京纽比特科技有限公司 )作为光源。器件的时间响应使用数字示波器(LeCroy Waverunner 8254)和被调制的375 nm LED(Thorlabs)作为光源来测试获得。光源的光强使用标准硅太阳能电池(PEC-S101,营口市站前区百科仪器耗材销售中心)或数字手持式光功率(PM100D,Thorlabs)和能量计表头(S120VC,Thorlabs)来标定。

\section{结果与讨论}

\subsection{\texorpdfstring{{\ce{CH3NH3PbCl3}}钙钛矿量子点材料物性表征}{PDFstring}}

\begin{figure}
\centering
  \includegraphics[width=\textwidth]{Fig_2.1.jpg}
  \caption{\rm \ce{MAPbCl3}钙钛矿量子点甲苯分散液和甲苯溶液(a)在白色荧光灯下以及(b)在365 nm波长的紫外灯下的光学照片。}
  \label{fig:2.1}
\end{figure}

\begin{figure}
\centering
  \includegraphics[width=\textwidth]{Fig_2.2.jpg}
  \caption{\rm (a) \ce{MAPbCl3}钙钛矿量子材料的HRTEM图片,(b) \ce{MAPbCl3}钙钛矿量子点XRD谱图,(c) \ce{MAPbCl3}钙钛矿量子点分散液、PTAA薄膜和氧化锌薄膜的吸收图谱,(d) 掺杂不同量\ce{MAPbCl3}钙钛矿量子点后PTAA薄膜的吸收图谱。}
  \label{fig:2.2}
\end{figure}

首先,我使用了Zhang等人之前报道过的“过饱和溶液重结晶”方法来制备\ce{MAPbCl3}钙钛矿量子点\cite{RN10}。 图\ref{fig:2.1}a和b分别展示了\ce{MAPbCl3}钙钛矿量子点分散液和甲苯溶液在白光和紫外光下的光学照片,表明\ce{MAPbCl3}钙钛矿量子点可以在甲苯中均匀分散。

通过使用HRTEM和XRD进一步表征验证确实得到了\ce{MAPbCl3}钙钛矿量子点:图\ref{fig:2.2}a中的HRTEM图像表明,所制备的钙钛矿量子点尺寸在10 nm左右;如图\ref{fig:2.2}b所示,在15.6\degree、22.1\degree、31.5\degree、35.3\degree、38.8\degree 和45.2\degree 的衍射峰分别对应于$\left\{\num{100}\right\}$、$\left\{\num{110}\right\}$、$\left\{\num{200}\right\}$、$\left\{\num{210}\right\}$、$\left\{\num{211}\right\}$和$\left\{\num{220}\right\}$晶面族,表明所制备的\ce{MAPbCl3}量子点为立方相钙钛矿矿晶体\cite{RN33}。我还测试了制作器件所需的三种主要材料的吸收光谱(图\ref{fig:2.2}c),说明\ce{MAPbCl3}钙钛矿量子点、PTAA薄膜和ZnO薄膜都主要吸收波长从300到420 nm之间的紫外光。然后将不同浓度的\ce{MAPbCl3}钙钛矿量子点掺杂到PTAA中,通过测试薄膜的吸收图谱来观察钙钛矿量子点的引入对有机半导体薄膜吸收光谱的影响。图\ref{fig:2.2}d展示了不掺杂和掺杂三种不同量\ce{MAPbCl3}钙钛矿量子点的PTAA薄膜的吸收光谱,四条谱线几乎重合,表明适量掺杂钙钛矿量子点对PTAA薄膜的光吸收性质影响甚微。

\subsection{\texorpdfstring{基于\ce{CH3NH3PbCl3}掺杂的紫外光电探测器性能测试及优化}{PDFstring}}%基于量子点掺杂的紫外光电探测器性能表征

在完成主要材料的物性表征后,我制作了基于\ce{MAPbCl3}钙钛矿量子点掺杂的PTAA薄膜的光电探测器。这种光电探测器的器件结构为ITO/ZnO/\ce{MAPbCl3}钙钛矿量子点掺杂PTAA/\ce{MoO_{\rm{x}}}/Ag(图\ref{fig:2.3}a)。

\begin{figure}[htb]
\centering
  \includegraphics[width=\textwidth]{Fig_2.3.jpg}
  \caption{\rm (a)紫外光电探测器器件结构示意图,(b)制作器件所用各材料能级示意图,(c)器件制作工艺优化前掺杂不同量\ce{MAPbCl3}钙钛矿量子点的器件暗态下\emph{J-V}曲线,(d)优化器件制作工艺后基于掺杂不同量\ce{MAPbCl3}钙钛矿量子点的紫外光电探测器暗态\emph{J-V}曲线(虚线)和光照下的\emph{J-V}曲线(实线,在光强为100\ \si{mW \ cm^{-2}}的氙灯照射下)。}
  \label{fig:2.3}
\end{figure}

为更好地理解器件中电荷地输运和抽取,将组成器件各材料的能级示意图展示在图\ref{fig:2.3}b中。从图中可以看出,具有合适能级的\ce{MAPbCl3}在PTAA和ZnO之间插入,可能会促进从PTAA中抽取电子和ZnO中空穴的注入,有利于器件各性能的提高。在后续实验中我系统地研究了\ce{MAPbCl3}钙钛矿量子点的引入对器件性能的影响。首先,探索了掺杂不同浓度\ce{MAPbCl3}钙钛矿量子点后器件暗电流密度的变化规律。如图\ref{fig:2.3}c所示,在初步研究中,所做器件使用了没有优化过制备工艺的ZnO薄膜,尽管在反向偏压下存在较大的漏电流以及较高的暗电流密度,但还是可以看出一定的规律——向PTAA中掺杂较低浓度\ce{MAPbCl3}钙钛矿量子点后,器件的暗电流密度变化程度不是很大;但是当掺杂浓度提高到0.5 wt\%的时候,器件的暗电流密度增加了大概一个数量级。因此,在后面的研究中,所用到的\ce{MAPbCl3} 量子点的掺杂浓度最高为0.25 wt\%。在优化了ZnO薄膜的制作工艺后,器件的暗态电流密度降低大概三个数量级,并且随着\ce{MAPbCl3}钙钛矿量子点掺杂浓度增加没有明显变化(图\ref{fig:2.3}d)。有趣的是,图\ref{fig:2.3}d表明了随着\ce{MAPbCl3}钙钛矿量子点掺杂浓度的增加器件的光电流密度得到了显著提升,说明这种掺杂方式可以有效提高器件各性能参数。

考虑到不同材料的能级结构,我还制作了使用其他材料作为传输层和具有不同器件结构的器件——使用\ce{SnO2}薄膜替换掉ZnO薄膜和使用反式p-n器件结构(ITO/\ce{NiO_{\rm{x}}}/\\掺杂不同量\ce{MAPbCl3}钙钛矿量子点的PTAA/LiF/C60/BCP/Cu)。然而,如同图\ref{fig:2.3.1}所示,这两种器件的暗电流密度都在相对较高的水平,因此,在后续的实验中,主要对基于ZnO薄膜的n-p结构的器件来展开研究。

\begin{figure}[htb]
\centering
  \includegraphics[width=\textwidth]{SnO2&NiOx.jpg}
  \caption{\rm (a)基于\ce{SnO2}薄膜作为电子传输层n-p型器件结构的光电探测器和(b)基于\ce{NiO_{\rm{x}}}作为空穴传输层p-n型器件结构的光电探测器在暗态下\emph{J-V}曲线(虚线)和光照下\emph{J-V}曲线(实线,在光强为100\ \si{mW \ cm^{-2}}的氙灯照射下)。}
  \label{fig:2.3.1}
\end{figure}

$R$是光电探测器的一个重要参数,它可以用公式\ref{eq-1.3.1}来计算。如图\ref{fig:2.4}a所示,在偏压为0 V时,基于未掺杂\ce{MAPbCl3}钙钛矿量子点的PTAA薄膜的光电探测器在390 nm光照下的响应度为{7.08} \si{mA\ W^{-1}},与之前报道的数值大小相近\cite{RN25};而基于掺杂了\ce{MAPbCl3}钙钛矿量子点的PTAA薄膜的光电探测器在同样光照条件下的$R$几乎翻倍,达到了{12.27} \si{mA\ W^{-1}}。进一步测试得到了器件随时间变化的暗电流密度,并通过FFT得到了器件的噪声电流密度和频率的关系,如图\ref{fig:2.4}b所示,未掺杂\ce{MAPbCl3}钙钛矿量子点和掺杂了\ce{MAPbCl3}钙钛矿量子点器件的噪声密度分别为$\approx 5.46$和$\approx 6.45\ \si{fA \ Hz^{-1/2}}$,都在非常低的水平。得到了$i_{\rm{n}}$和$R$,就可以通过公式\ref{eq-1.3.3}来计算器件的$D^*$。图\ref{fig:2.4}a表明,未掺杂\ce{MAPbCl3}钙钛矿量子点和掺杂了\ce{MAPbCl3}钙钛矿量子点器件$D^*$最大值分别达到了\num{4.05e11}\ Jones和\num{6.07e11}\ Jones,这说明了在\ce{MAPbCl3}钙钛矿量子点辅助下,器件的灵敏度也得到了提高。

\begin{figure}[htb]
\centering
  \includegraphics[width=\textwidth]{Fig_2.4.jpg}
  \caption{\rm (a) 比较掺杂与未掺杂\ce{MAPbCl3}钙钛矿量子点紫外光电探测器在偏压为0\ \si{V}下的光谱响应度与比探测度和 (b) 噪声密度图谱 。}
  \label{fig:2.4}
\end{figure}

我还研究了ZnO薄膜的厚度对器件性能的影响,首先,通过重复旋涂醋酸锌前驱体来增加ZnO薄膜的厚度,如同图\ref{fig:2.4.1}a所示,一层ZnO薄膜的厚度大约为60 nm,ZnO薄膜的厚度随旋涂次数增加而基本上呈线性增加,在旋涂四层后,就得到厚度约为220 nm ZnO薄膜。然后,测试了基于不同厚度ZnO薄膜的器件在0 V偏压下的响应度(图\ref{fig:2.4.1}b),发现随着ZnO薄膜厚度的增加,器件的响应度图谱的半高宽在减小,从61 nm(基于一层ZnO薄膜的器件)减小至41 nm(基于四层ZnO薄膜器件),这一实验结果之前报道的现象相符\cite{RN25}。对于ZnO/PTAA结构的器件,器件中产生的载流子可以被分为两部分:一部分由短波长紫外光在ZnO薄膜中产生,另一部分由长波长紫外光在PTAA薄膜中产生。一般来说,在ZnO薄膜中电荷产生率较低,在ZnO薄膜中产生的空穴需要穿过ZnO薄膜和PTAA薄膜,因此,随着ZnO薄膜厚度的增加,在ZnO薄膜中产生的空穴需要更长的电荷传输距离,较长的电荷传输距离和ZnO较低的空穴载流子迁移率可能会导致严重的载流子复合,从而导致被ZnO薄膜吸收的能量较高的光子的量子效率被减弱。而对于PTAA薄膜来说,其电荷产生率是ZnO中电荷产生率的五倍以上,并且大部分载流子都是在靠近电极处产生的,具有较高的载流子抽取效率,因此,被PTAA薄膜吸收的能量稍低的光子的量子效率较高。根据以上解释,可以通过调控ZnO薄膜的厚度来调节光电探测器的光谱响应范围。%这也就解释了为什么随着ZnO薄膜厚度的增加器件的响应度图谱的半高宽逐渐减小。

\begin{figure}[htb]
\centering
  \includegraphics[width=\textwidth]{increase thickness of ZnO-1.jpg}
  \caption{\rm (a) ZnO薄膜的厚度随着旋涂层数的变化(b) 在0 V偏压下器件的归一化光谱响应度随着ZnO薄膜厚度的变化。}
  \label{fig:2.4.1}
\end{figure}


\subsection{测试PTAA薄膜的空穴迁移率}

\begin{figure}[htb]
\centering
  \includegraphics[width=\textwidth]{SCLC&timeresponse-1.jpg}
  \caption{\rm (a) 使用SCLC测试PTAA薄膜空穴迁移率的器件结构示意图和(b)器件的\emph{J-V}曲线,(c) 掺杂与未掺杂\ce{MAPbCl3}钙钛矿量子点的PTAA薄膜在不同场强下的空穴迁移率,(d) 基于掺杂与未掺杂\ce{MAPbCl3}钙钛矿量子点的PTAA的紫外光电探测器时间响应图谱(在偏压为0 \si{V}下测得)。}
  \label{fig:2.5}
\end{figure}

为研究掺杂\ce{MAPbCl3}钙钛矿量子点对PTAA层载流子输运性能的影响,使用了空间电荷限制电流(space charge limited current,SCLC)模型,并采用图\ref{fig:2.5}a所示的器件结构,来分析未掺杂和掺杂\ce{MAPbCl3}钙钛矿量子点的PTAA薄膜的空穴迁移率。从\ref{fig:2.5}b所示的0-10 V 暗态下测得的\emph{J-V}曲线,可以看出PTAA薄膜在掺杂\ce{MAPbCl3}钙钛矿量子点后变得更加导电,说明其载流子迁移率或者载流子的密度得到了提高。使用Mott-Gurney方程来拟合\emph{J-V}曲线:

\begin{equation}\label{eq-2.3}
{J = \frac{{9}\varepsilon_{\rm{0}}\varepsilon_{\rm{r}}\mu V^2}{8L^3}}
\end{equation}

其中$\varepsilon_{\rm{0}}$ 真空中的介电常数,$\varepsilon_{\rm{r}}$ 是介电常数(对PTAA这种材料来说,$\varepsilon_{\rm{r}} \approx 3$),$L$ 为薄膜的厚度($\approx$ 155\ nm),$\mu$ 为空穴的迁移率。此外,对有机半导体来说,其载流子的迁移率和所加电场强度$E$有关,可使用Pool-Frenkel模型\cite{RN5}来描述

\begin{equation}\label{eq-2.4}
\mu = \mu_{\rm{0}}exp(\gamma\sqrt{E})
\end{equation}

公式\ref{eq-2.4} 中$\mu_{\rm{0}}$为无电场下的迁移率,$\gamma$为材料有关的系数。结合公式\ref{eq-2.3} 和\ref{eq-2.4} 可得到

\begin{equation}\label{eq-2.5}
J = \frac{9\varepsilon_{\rm{0}}\varepsilon_{\rm{r}}\mu_0E^2exp(\gamma\sqrt{E})}{8L}
\end{equation}

使用公式\ref{eq-2.5}和图\ref{fig:2.5}b中的数据,可计算得出$\mu_0$和$\gamma$,然后得到了如图\ref{fig:2.5}c所示未掺杂和掺杂\ce{MAPbCl3}钙钛矿量子点PTAA薄膜和场强有关的空穴迁移率。掺杂了\ce{MAPbCl3}钙钛矿量子点的PTAA薄膜空穴迁移率$\approx$ \num{1.0e-4}\ \si{cm^2\ V^{-1}\ s^{-1}}(在电场强度\num{1.0e4} 与\num{1.0e5}\ \si{V \ cm^{-1}} 之间的平均值),大约为未掺杂的PTAA薄膜空穴迁移率($\approx$ \num{1.5e-5}\ \si{cm^2\ V^{-1}\ s^{-1}})的6倍。得以提升的空穴迁移率可能会促进基于PTAA薄膜的光电二极管中电荷的输运和抽取,进而提高了器件性能,这与之前的实验结果相吻合。为进一步验证掺杂钙钛矿量子点的效果,还比较了基于掺杂与未掺杂\ce{MAPbCl3}钙钛矿量子点的PTAA薄膜的器件的响应速度。如图\ref{fig:2.5}d 所示,在掺杂了钙钛矿量子点之后,器件的响应时间从67\ μs降低到了28\ s,这一结果再次验证了掺杂\ce{MAPbCl3}钙钛矿量子点可以提高PTAA薄膜的载流子输运性能,进而提高了器件的响应速度。

\subsection{柔性器件}

基于重量轻、可弯折衬底的柔性光电探测器因具有在大面积可折叠显示、可穿戴设备和航空航天等方面上的巨大应用潜力而备受关注\cite{RN184}。考虑到此项工作中使用了有机半导体材料作为器件的主要活性层,我又尝试了在PET柔性导电衬底上制作了和图\ref{fig:2.3}a结构相同的光电探测器。

\begin{figure}[ht]
\centering
  \includegraphics[width=\textwidth]{Fig_2.6.jpg}
  \caption{\rm (a) 柔性器件在不同弯折角度下的照片,(b) 柔性器件在波长为395 nm LED灯珠(0.83\ \si{mW\ cm^{-2}})照射下多次弯折后的时间响应图谱(在偏压为0 \si{V}下测得)。}
  \label{fig:2.6}
\end{figure}

为了研究器件的柔韧性,把器件按照图\ref{fig:2.6}a那样多次弯折,并测试了器件在不同弯折次数后的光电流密度随时间的的变化。从图\ref{fig:2.6}b可以看出,器件经过多次弯折后在调制的光照下仍然就有较快的响应速度,并且器件的光电流密度随弯折次数的增加下降程度较小。这些重复性较好的性能参数,证明了可将此种结构应用于柔性器件。

\subsection{器件稳定性以及尝试掺杂全无机钙钛矿量子点}

考虑到有机无机杂化钙钛矿材料较差的稳定性,我还研究了器件的长期稳定性,通过测试器件在放置不同时间后的光电流变化来表征器件的长期稳定性如何。

\begin{figure}[htb]
\centering
  \includegraphics[width=\textwidth]{Fig_2.7.jpg}
  \caption{\rm (a) 基于掺杂\ce{MAPbCl3}钙钛矿量子点的PTAA薄膜的器件的长期稳定性测试图谱(器件被存放在手套箱中并在温度$\approx$ 25℃和湿度$\approx$ 60{\%}RH 条件下,在光强为100 \si{mW\ cm^{-2}}下测试器件的光电流),(b) 基于掺杂与不掺杂CsPbCl$_3$钙钛矿量子点的PTAA薄膜的器件在光强为100 \si{mW\ cm^{-2}}氙灯照射下的\emph{J-V}曲线,。}
  \label{fig:2.7}
\end{figure}

如图\ref{fig:2.7}a所示,器件在放置2个月后其光电流依旧为初始光电流的82\%,显示出了较好的长期稳定性。此外,我还进一步尝试了将全无机\ce{CsPbCl3}钙钛矿量子点掺杂到PTAA薄膜中去,期望在提高器件光电性能的同时也可以提高器件的稳定性。但是,如同图\ref{fig:2.7} b展示的那样,将全无机\ce{CsPbCl3}钙钛矿量子点掺杂到PTAA薄膜中后,器件的光电流密度并未发生明显的变化。

%如\ref{fig:2.7} a所示,实验表明,掺杂了CsPbCl$_3$量子点的器件较未掺杂CsPbCl$_3$量子点的器件光电流密度并未明显变化。进一步实验表明,即使使用了CH$_3$NH$_3$PbCl$_3$量子点,器件稳定性也不是很差。如\ref{fig:2.7} b所示,基于PTAA掺杂CH$_3$NH$_3$PbCl$_3$量子点的器件,在两个月后器件的光电流仍为初始值的82\%,显示出了较好的稳定性。

\section{本章小结}

通过实验证明,将\ce{MAPbCl3}钙钛矿量子点掺杂到PTAA薄膜中,是一种可以提高基于ZnO/PTAA异质结紫外光电探测器性能的简单、新颖的策略。这种光电探测器可使用溶液法制备,具有极低的暗电流、极低噪声电流和良好的稳定性。通过掺杂适量\ce{MAPbCl3}钙钛矿量子点,在几乎不改变PTAA层对光的吸收的基础上可以提高器件的响应度和比探测度。使用SCLC研究了未掺杂和掺杂\ce{MAPbCl3}钙钛矿量子点PTAA薄膜的电荷输运性质,发现掺杂了\ce{MAPbCl3}钙钛矿量子点PTAA薄膜的空穴迁移率$\approx$ \num{1.0e-4}\ \si{cm^2\ V^{-1}\ s^{-1}},大约为未掺杂的PTAA薄膜空穴迁移率的6倍。通过测试两种类型器件的响应时间,发现掺杂钙钛矿量子点这种策略可以有效提高器件的响应速度,再次验证了掺杂可提高材料的载流子迁移率。此外,还制作了柔性器件并验证了器件的柔韧性。最后,尝试了将有机无机杂化\ce{MAPbCl3}钙钛矿量子点替换为全无机\ce{CsPbCl3}钙钛矿量子点,期望在提高器件性能的同时可以进一步提高器件的长期稳定性,但是,实验结果表明,将全无机\ce{CsPbCl3}钙钛矿量子点掺杂到PTAA薄膜中后,器件的光电流密度并未发生明显的变化。
%实验证明将钙钛矿量子点掺杂到PTAA中,是一种有效提高有机半导体载流子迁移率的策略。掺杂使得有机半导体的载流子迁移率得到了提高,进而提高了器件诸如,响应度、探测度和响应速度等器件性能参数。

%%%=== 基于热蒸发法制备的宽带隙钙钛矿紫外光电探测器 ========%%%

\chapter{基于热蒸发法制备的全无机钙钛矿在紫外光电探测器中的应用研究}

\section{引言}

氯化物钙钛矿具有较大的带隙宽度以及卓越的光电性能,可用于紫外光检测。在有关金属卤化物钙钛矿在紫外光电探测器上的应用研究中,大多数被报道的器件都是基于\ce{MAPbCl3}有机无机杂化钙钛矿的。虽然这些器件都取得了较好的光电性能,但是有机无机杂化钙钛矿材料中的有机离子存在着不可逆转的挥发和分解\cite{RN81,RN83}以及外电场作用下移动\cite{RN84,RN85}等问题,使得有机无机杂化钙钛矿材料稳定性较差,导致基于有机无机杂化钙钛矿材料光电器件寿命较短,这些问题在一定程度上阻碍了钙钛矿材料的实际应用。

众多研究表明,\ce{CsPbX3}(X = I,Br,Cl)全无机钙钛矿材料比有机无机杂化钙钛矿材料的稳定性要好很多\cite{RN89,RN87,RN88}。\ce{CsPbCl3}全无机钙钛矿是一种具有良好光电性能的宽带隙半导体材料,其主要吸收波长为400 nm之前的光线,因此适合用于制作高性能紫外光电探测器。但是,用于制备\ce{CsPbCl3}全无机钙钛矿薄膜的前驱体CsCl在常规的DMF、DMSO等有机溶剂中溶解度较低\cite{RN137,RN89,RN90},导致使用常规溶液法旋涂制备\ce{CsPbCl3}全无机钙钛矿薄膜变得十分困难,因而基于\ce{CsPbCl3}全无机钙钛矿薄膜的紫外光电探测器较少被报道。

钙钛矿材料在诸如太阳能电池、发光二极管以及光电探测器等光电器件上的广泛应用,除了钙钛矿本身所具有的优异光电特性,也离不开制备钙钛矿材料的工艺的不断发展。其中,热蒸发法从原理上来说是一种可以直接沉积各种类型高质量钙钛矿薄膜的方法。这种不受材料溶解度限制的物理气相沉积法为制备\ce{CsPbCl3}全无机钙钛矿薄膜提供了一种可行的思路。最近,Yang等人就使用了热蒸发法制备了\ce{CsPbCl3}全无机钙钛矿薄膜,并将其应用到光电二极管型紫外光电探测器\cite{RN90}。在对沉积的\ce{CsPbCl3}全无机钙钛矿薄膜120℃退火处理后,制作的器件展示了优异的器件性能和较好的稳定性。

为进一步探索\ce{CsPbCl3}全无机钙钛矿薄膜制备工艺,以及将\ce{CsPbCl3}全无机钙钛矿薄膜应用于紫外光电探测器时所需的后处理工艺和器件结构的选择与优化,在这项工作中,我尝试了类似的热蒸发法工艺来制备\ce{CsPbCl3}全无机钙钛矿薄膜,使用了吸收光谱、XRD和SEM等表征手段来观察不同退火温度下薄膜形貌和结晶程度的变化,并且优化了将\ce{CsPbCl3}全无机钙钛矿薄膜应用于紫外光电探测器的器件结构。将\ce{CsPbCl3}全无机钙钛矿薄膜应用于紫外光电探测器,器件的EQE随温度的变化趋势和薄膜的吸收光谱、XRD图谱以及SEM图谱随温度变化趋势相符,但在较高退火温度下,器件的性能出现了明显下降。根据在ZnO上沉积钙钛矿薄膜的研究,推测出一种可能导致器件性能下降的机制。这项工作为将\ce{CsPbCl3}全无机钙钛矿薄膜应用于紫外光电探测器等光电器件奠定了一定的实验研究基础。%由于钙钛矿材料本身优良的光电特性,使其在太阳能电池、发光二极管以及光电探测器上有着广泛的应用。虽然最近几年钙钛矿材料与器件方面的研究取得了丰硕的成果,但是钙钛矿材料存在稳定性较差的问题,在一定程度上阻碍了钙钛矿材料的实际应用。众多研究表明,全无机钙钛矿材料比有机无机杂化钙钛矿材料的稳定性要好很多。 []举一些文献提及全无机钙钛矿稳定性较好的实验。但是全无机钙钛矿的前驱体,尤其是CsCl

\section{实验}

\subsection{材料}

聚[双(4-苯基)(2,4,6-三甲基苯基)胺](PTAA,M$_{\rm{n}}$ = 15,000\textasciitilde  25,000)、氯化铅(\ce{PbCl2},>99.99\%)和浴铜灵(BCP,>99\%)购买于西安宝莱特光电科技有限公司。氯化铯(\ce{CsCl},99\%)、乙醇胺(\ce{NH2CH2CH2OH},99\%)和氯苯(AR,99\%)购买于阿拉丁试剂。醋酸锌(\ce{Zn(CH3COO)2},99\%)购买于九鼎化学。三氧化钼(\ce{MoO_\rm x},99.99\%)购买于上海易势化工。甲苯(AR)购买于国药集团化学试剂有限公司。乙二醇单甲醚(\ce{CH3OCH2CH2OH},99\%)购买于泰坦。聚[双(4-苯基)(4-丁基苯基)胺](Poly-TPD,M$_{\rm{w}}$ > 10,000)购买于台湾机光科技股份有限公司。碳60(C60,99.5\%)购买于南京先丰纳米材料科技有限公司。

\subsection{器件制作}
所有器件都是在商业定制图案化的氧化铟锡(ITO)玻璃衬底上制作的。先使用洗洁精清洗ITO玻璃衬底,然后使用丙酮和乙醇分别超声清洗10分钟。清洗干净的衬底使用压缩空气吹干并放置在臭氧清洗机中处理10分钟后备用,然后按照ITO/Poly-TPD/\ce{CsPbCl3}/C60/BCP/Cu或ITO/ZnO/\ce{CsPbCl3}/PTAA/\ce{MoO_{\rm{x}}}/Ag的器件结构在衬底上依次沉积这些薄膜。ZnO前驱体溶液按照之前报道的方法来制备\cite{RN1}。以转速为2000 rpm、旋涂时长为15 s的旋涂程序旋涂ZnO前驱体溶液来制备薄膜,再将旋涂得到的薄膜放在热台上顺序进行80\textcelsius\ (10分钟)、140\textcelsius\ (10分钟)和200\textcelsius\ (20分钟)的退火后就得到了ZnO薄膜。将Poly-TPD溶解在氯苯中,配置浓度为2 \si{mg\ mL^{-1}}的溶液。以2000 rpm、旋涂时长为30 s的旋涂程序来旋涂Poly-TPD溶液来制备薄膜,将旋涂制备的薄膜放置在100\textcelsius 的热台上退火10分钟后就可以得到Poly-TPD薄膜。再将这些沉积了ZnO薄膜或Poly-TPD薄膜的衬底转移到真空镀膜机(水和氧含量都小于0.1 ppm)中沉积其他层和电极。使用热蒸发的方法沉积\ce{CsPbCl3}薄膜,所使用的设备由武汉普迪真空科技有限公司定制,待热蒸发设备腔体的气压降到\num{5e-4} \si{Pa}以下,打开束源炉的加热电源和对应盖板,\ce{PbCl2}和\ce{CsCl}放置在石英坩埚中分别用两个束源炉来加热。使用频率计来检测束源炉中药品的蒸发速率,通过调节束源炉的温度来调整前驱体的蒸发速率,待\ce{PbCl2}的蒸发速率稳定在1 - 1.2 Hz\ $\si{s^{-1}}$时,开启\ce{CsCl}对应束源炉的加热电源,缓慢调节使其蒸发速率稳定在0.6 - 0.8 Hz\ $\si{s^{-1}}$,随后开启基片挡板开始沉积\ce{CsPbCl3}薄膜,在频率计数值变化为7900 Hz后(对应薄膜厚度大约为200 nm),关闭基片挡板,然后待热蒸发设备关闭后取出沉积的\ce{CsPbCl3}薄膜,最后将\ce{CsPbCl3}薄膜在不同温度下退火处理30分钟后继续沉积其他薄膜层。将甲苯溶液加入到PTAA粉末中配置成浓度为1.5 \si{mg\ mL^{-1}}的溶液,以150 rpm搅拌1小时以备用。使用6000 rpm、旋涂时长为30 s的旋涂程序来旋涂PTAA溶液来制备薄膜,将旋涂制备的PTAA薄膜放置在100\textcelsius 的热台上退火10分钟。最后使用热蒸发的方法将\ce{MoO_{\rm{x}}}(7.5 nm)和Ag(60 nm)依次沉积在PTAA薄膜上,同样使用热蒸发的方式按照C60(30 nm)、BCP(3.2 nm)和Cu(60 nm)的顺序沉积在\ce{CsPbCl3}薄膜上,最终所制作的器件的有效面积为0.1 \si{cm^2}。

\subsection{测试与表征}

使用FR-Basic-UV/NIR-HR spectroscope (ThetaMetrisis)来测试材料的吸收光谱。扫描电子显微镜照片(SEM)使用Zeiss Sigma场发射扫描电子显微镜拍摄。材料的结构使用X射线衍射(XRD)来表征,所使用的仪器为D8(Cu K$\upalpha$,$\uplambda$ = 1.5418 \ {\AA})X射线仪,扫描范围从10{\degree}到60{\degree}。薄膜的厚度使用台阶仪(Alpha-step D-600 profile meter,KLA-Tencor)来获得。\emph{J-V}特性曲线(包括暗态下和光照下的)使用半导体分析仪(B1500A)在室温下测试获取。在测试光照下的\emph{J-V}特性曲线时,使用紫外光固化机(UVPL-4C)上波长为395 nm的照射头作为光源,光源光强为2.67\ \si{mW \ cm^{-2}}。光源的光强使用数字手持式光功率(PM100D,Thorlabs)和能量计表头(S120VC,Thorlabs)来标定。

\section{结果与讨论}

\subsection{\texorpdfstring{{\ce{CsPbCl3}}全无机钙钛矿薄膜的制备及物性表征}{PDFstring}}

如\ref{fig:3.1}a所示,使用热蒸发工艺制备\ce{CsPbCl3}全无机钙钛矿薄膜时用到了两个束源炉,两个石英坩埚中分别装有\ce{PbCl2}和CsCl粉末。通过调节两个束源炉的温度来调整\ce{PbCl2}和CsCl的蒸发速率,当两者速率为合适比例时就可得到\ce{CsPbCl3}全无机钙钛矿薄膜。此外,在热蒸发的同时通过旋转放置衬底的支架使薄膜沉积得更加均匀。图\ref{fig:3.1}b为使用热蒸发法沉积的\ce{CsPbCl3}全无机钙钛矿薄膜制作紫外光电探测器的最后一步工艺。可以看出,通过热蒸发工艺制备的\ce{CsPbCl3}全无机钙钛矿薄膜几乎是透明的,这和其较宽的带隙相符。

\begin{figure}[htb]
\centering
  \includegraphics[width=\textwidth]{Fig_3.1.jpg}
  \caption{\rm (a)使用热蒸发法制备\ce{CsPbCl3}全无机钙钛矿薄膜的示意图,(b)将热蒸发法制备的\ce{CsPbCl3}全无机钙钛矿薄膜用于制作器件,在蒸镀Ag电极之后的照片。}
  \label{fig:3.1}
\end{figure}

为确定\ce{PbCl2}和CsCl的最优蒸发速率,使用\ce{PbCl2}和\ce{CsCl}的密度以及对应的相对分子质量与薄膜厚度的关系来调节二者的蒸发速率,以获得两者摩尔比为1:1的\ce{CsPbCl3}全无机钙钛矿薄膜。考虑到\ce{PbCl2}(相对分子质量为278)和\ce{CsCl}(相对分子质量为168.5)的密度分别为5.850 和3.988 \si{g \ cm^{-3}},在将\ce{PbCl2}和\ce{CsCl}的蒸发速率控制在1.0 - 1.2 Hz\ $\si{s^{-1}}$和0.6 - 0.8 Hz\ $\si{s^{-1}}$的范围时,就可得到如图\ref{fig:3.2.0}a所示的XRD图谱,15.8{\degree}、31.9{\degree}和48.7{\degree}处的衍射峰分别对应$\left\{\num{100}\right\}$、$\left\{\num{200}\right\}$和$\left\{\num{300}\right\}$晶面族,显示出较好的立方相\cite{RN137,RN101,RN119}。但是,如果蒸发速率稍加快或\ce{PbCl2}与\ce{CsCl}蒸发比例发生变化时,就有可能会出现杂相(图\ref{fig:3.2.0}b)。因此,后续实验中,都将\ce{PbCl2}和\ce{CsCl}的蒸发速率控制在1.0 - 1.2 Hz\ $\si{s^{-1}}$和0.6 - 0.8 Hz\ $\si{s^{-1}}$的范围。

\begin{figure}[ht]
\centering
  \includegraphics[width=\textwidth]{Fig_3.2.1-0-1.jpg}
  \caption{\rm (a) \ce{PbCl2}与\ce{CsCl}的蒸发速率分别为1.0 - 1.2 Hz\ $\si{s^{-1}}$与0.6 - 0.8 Hz\ $\si{s^{-1}}$和(b) \ce{PbCl2}和\ce{CsCl}的蒸发速率分别为1.6 Hz\ $\si{s^{-1}}$与1.0 Hz\ $\si{s^{-1}}$时所制备薄膜的XRD图谱。}
  \label{fig:3.2.0}
\end{figure}

\iffalse 如\ref{fig:3.2} 所示,吸收图谱和XRD图谱综合表明得到了CsPbCl$_3$钙钛矿材料,退火可使薄膜结晶性变好。\fi
在通过热蒸发的工艺获得\ce{CsPbCl3}全无机钙钛矿薄膜后,我继续尝试了在不同温度下对薄膜进行退火处理,并使用吸收光谱、XRD和SEM来观察退火温度对薄膜形貌以及结晶程度的影响。如图\ref{fig:3.2}所示的吸收光谱,\ce{CsPbCl3}全无机钙钛矿薄膜主要吸收波长为420 nm之前的光,对应带隙$\approx$ 2.9 eV,再次说明了这种材料适合用于制备紫外光电探测器,也解释了为什么图\ref{fig:3.1}b中的钙钛矿薄膜几乎是透明的。随着退火温度的增加,\ce{CsPbCl3}全无机钙钛矿薄膜对波长约为390 nm之前的光吸收有增加的趋势,这可能会导致器件的光电流密度有所上升。此外,随着退火温度的增加,薄膜吸收光谱的吸收边有向长波长方向移动的趋势,这可能与随着退火温度增加\ce{CsPbCl3}钙钛矿晶粒长大导致薄膜表面粗糙度增加而增强了薄膜的光学散射有关\cite{RN91}。

\begin{figure}[htb]
\centering
  \includegraphics[width=0.5\textwidth]{Fig_3.2-1.jpg}
  \caption{\rm 经过不同退火温度处理后的\ce{CsPbCl3}全无机钙钛矿薄膜的吸收图谱。}
  \label{fig:3.2}
\end{figure}

进一步地,我测试了不同退火温度下基于\ce{CsPbCl3}全无机钙钛矿薄膜的器件的XRD图谱,如图\ref{fig:3.2.1}a所示,15.8{\degree}和31.9{\degree}的衍射峰分别对应$\left\{\num{100}\right\}$和$\left\{\num{200}\right\}$晶面族,再次说明所制备的\ce{CsPbCl3}全无机薄膜为立方相钙钛矿,并且随着退火温度的升高钙钛矿未发生相变。通过分析图\ref{fig:3.2.1}a中的数据,得出如图\ref{fig:3.2.1}b所示的200衍射峰半高宽(弧度)与退火温度的关系,可以看出\ce{CsPbCl3}全无机钙钛矿薄膜在31.9{\degree}处衍射峰的半高宽随着退火温度的增加有减小的趋势,说明了随着退火温度的增加\ce{CsPbCl3}全无机钙钛矿薄膜中发生了再结晶过程并使晶体取向性变得更好\cite{RN106}。\iffalse cite{X-Ray Diffraction A Practical Approach} \fi

\begin{figure}[ht]
\centering
  \includegraphics[width=\textwidth]{Fig_3.2.1.jpg}
  \caption{\rm (a) 基于经过不同温度退火处理后的\ce{CsPbCl3}全无机钙钛矿薄膜的器件的XRD图谱, (b) $\left\{\num{200}\right\}$晶面族衍射峰半高宽(弧度)及其倒数随退火温度的变化。}
  \label{fig:3.2.1}
\end{figure}

\begin{figure}[htbp]
\centering
  \includegraphics[width=\textwidth]{Fig_3.3-2.jpg}
  \caption{\rm 经过不同温度退火处理后的\ce{CsPbCl3}全无机钙钛矿薄膜的SEM照片。}
  \label{fig:3.3}
\end{figure}

为了更直观地观察不同退火温度对\ce{CsPbCl3}全无机钙钛矿薄膜形貌的影响,我还测试了薄膜的SEM图。通过图\ref{fig:3.3}可以看出,退火处理明显地改变了薄膜的形貌:未经过后退火处理的\ce{CsPbCl3}全无机钙钛矿薄膜晶粒尺寸较小(50 nm左右);使用120℃的温度对薄膜退火处理30分钟后,薄膜的形貌出现变化,晶粒稍增大;在进一步提高退火处理的温度至150℃后,薄膜的形貌出现明显变化,晶粒继续长大,尺寸超过200 nm;当退火温度上升至200℃时,晶粒尺寸达到400-500 nm;继续增加退火处理的温度,薄膜形貌仍有变化,虽然晶粒尺寸不再明显增加,但未出现晶粒尺寸减小的趋势。总之,对\ce{CsPbCl3}全无机钙钛矿薄膜进行后退火处理,随着退火温度的增加薄膜中晶粒尺寸存在增大的趋势,这与解释薄膜吸收光谱的吸收边向长波长方向移动的猜想相符。值得注意的是,这一实验结果与Yang等人\cite{RN90}所报道的现象略有不同,他们在对薄膜进行退火处理后,薄膜中晶粒的尺寸随着温度增加先变大后减小,而我所制备的钙钛矿薄膜在较高退火温度下并未出现晶粒尺寸减小的迹象。我并没有进一步增加退火温度,因为发现将经过高温(250℃)退火处理后的薄膜应用于紫外光电探测器时,器件性能出现了明显的下降。

\subsection[基于{\ce{CsPbCl3}}的紫外光电探测器性能测试及优化]{\texorpdfstring{基于{\ce{CsPbCl3}}全无机钙钛矿薄膜的紫外光电探测器性能测试及优化}{PDFstring}}

随着退火温度的增加,\ce{CsPbCl3}全无机钙钛矿薄膜晶粒尺寸增大以及及结晶性变好,这会减少薄膜中晶界处的缺陷数量,因而我推测基于\ce{CsPbCl3}全无机钙钛矿薄膜的紫外探测器性能也会存在上升的趋势。为印证猜想,我制作了一种结构如同图\ref{fig:3.4}a所示的紫外光电探测器,从器件对应的SEM截面图可以看出蒸发制备的\ce{CsPbCl3}全无机钙钛矿薄膜厚度大约为200 nm。为了更好地理解所制备紫外光电探测器中电荷的运输和抽取,将制作器件所用各材料的能级示意图绘制在图\ref{fig:3.4}b中,可以看出,所选用的ZnO电子传输层和PTAA空穴传输层都有利于在光照下器件中载流子的传输。

\begin{figure}[ht]
\centering
  \includegraphics[width=\textwidth]{Fig_3.5.jpg}
  \caption{\rm  (a) 基于\ce{CsPbCl3}全无机钙钛矿薄膜的紫外光电探测器的器件结构示意图和对应的SEM截面图,(b) 组成紫外光电探测器各材料能级示意图。}
  \label{fig:3.4}
\end{figure}

\begin{figure}[ht]
\centering
  \includegraphics[width=\textwidth]{Fig_3.4.1-1.jpg} %0.5\textwidth可调节图片的比例,Lihao Cui,2021.2.5
  \caption{\rm (a) 器件结构为ITO/ZnO/\ce{CsPbCl3}/PTAA/\ce{MoO_{\rm{x}}}/Ag和ITO/ZnO/\ce{CsPbCl3}/\ce{MoO_{\rm{x}}}/Ag的紫外光电探测器的暗态\emph{J-V}曲线(虚线)与光照下的\emph{J-V}曲线(实线),(b) C60薄膜和\ce{CsPbCl3}薄膜的吸收光谱以及器件结构为ITO/ZnO/\ce{CsPbCl3}/PTAA/\ce{MoO_{\rm{x}}}/Ag和ITO/Poly-TPD/\ce{CsPbCl3}/C60/BCP/Cu的探测器在不同波长下的EQE。}
  \label{fig:3.4.1}
\end{figure}

为了简化制作器件的工艺,最初我还尝试了没有PTAA层的器件。通过测试这种类型的器件,发现器件的暗电流密度较高,如图\ref{fig:3.4.1}a所示的\emph{J-V}曲线。在使用了PTAA层后,器件的暗电流密度降低了大概两个数量级且光电流密度降低程度较小,说明PTAA层对负偏压下电子注入起到了一定的阻碍作用。此外,我还尝试制作了结构为ITO/Poly-TPD/\ce{CsPbCl3}/C60/BCP/Cu的器件,发现基于这种结构的器件EQE较高,并对可见光有一定的响应。从材料的吸收光谱可以看出C60薄膜对器件EQE的增加和可见光的响应起到了关键作用。因此,考虑到C60对可见光的吸收、有机物Poly-TPD较低的玻璃化转变温度(180℃)\cite{RN187}和PTAA对电子注入的阻挡作用,后续所做器件都采用了ITO/ZnO/\ce{CsPbCl3}/PTAA/\ce{MoO_{\rm{x}}}/Ag这种器件结构。

\begin{figure}[ht]
\centering
  \includegraphics[width=\textwidth]{Fig_3.6.jpg}
  \caption{\rm (a) 将经过不同温度退火处理后的\ce{CsPbCl3}全无机钙钛矿薄膜应用于紫外光电探测器,器件的暗态\emph{J-V}曲线(虚线)和光照下的\emph{J-V}曲线(实线),(b) 同一批器件EQE随处理\ce{CsPbCl3}全无机钙钛矿薄膜温度变化的{\iffalse 平均值加标准误(SE)\fi}统计箱图。}
  \label{fig:3.6}
\end{figure}

将经过不同温度后退火处理的\ce{CsPbCl3}全无机钙钛矿薄膜应用于紫外光电探测器,并测试了器件的\emph{J-V}曲线。从图\ref{fig:3.6}a可以看出器件在负偏压下的光暗电流密度都有随着退火温度的增加而上升趋势,将退火温度提高至220℃时暗电流密度变化程度不是很大,但是当温度上升至250℃时器件的暗电流密度出现骤增。基于在ZnO上沉积全无机钙钛矿薄膜的研究,推测出一种可能导致器件暗电流密度骤增的机制——醋酸根离子\ce{CH3COO^-}(\ce{Ac^-})与\ce{Cs^+}相互作用可减少\ce{CsPbCl3}全无机钙钛矿薄膜中缺陷,而随着退火温度的增加,ZnO薄膜中的\ce{Ac^-}在高温下挥发而减少,使得Cs-Ac之间的相互作用减弱,从而导致器件的暗电流密度增加\cite{RN93}。为更加直观地得出器件性能与处理\ce{CsPbCl3}全无机钙钛矿薄膜温度之间的关系,又考虑到EQE与$R$、$D^*$等性能参数都有密切的联系,因此,采用
EQE来探索这一变化规律。使用公式\ref{eq-1.3.1}和公式\ref{eq-1.3.2}共同来计算器件的EQE:

\begin{equation}\label{eq-3.1}
{\rm{EQE} }=  \frac{J_{\rm{ph}}}{P_{\rm{in}}}\frac{hc}{q\lambda} = \frac{{J}-{J_{\rm{d}}}}{P_{\rm{in}}}\frac{hc}{q\lambda}
\end{equation}

从图\ref{fig:3.6}a可以得到器件在一定偏压下的光暗电流密度$J$和$J_{\rm{d}}$,测试光源的光强$P_{\rm{in}}$为2.67\ \si{mW \ cm^{-2}},测试光源的波长$\lambda$为395 nm。图\ref{fig:3.6}b展示了器件在偏压为-0.1 V下随处理\ce{CsPbCl3}全无机钙钛矿薄膜退火温度变化的EQE,可以看出,随着退火温度的增加,紫外光电探测器的EQE有明显的上升趋势,从未退火之前的不足5\%上升至20\%左右(220℃)。说明对\ce{CsPbCl3}全无机钙钛矿薄膜进行后退火处理,可以显著提高器件的光电性能,这一实验结果印证了使用后退火处理\ce{CsPbCl3}全无机钙钛矿薄膜可以提高基于此的紫外光电探测器性能的猜想。

\section{本章小结}

为将稳定性更好的全无机钙钛矿材料应用于紫外光电探测器,通过热蒸发这种不受材料溶解度限制的工艺来制备\ce{CsPbCl3}全无机钙钛矿薄膜,探索了沉积钙钛矿薄膜时较为合适的蒸发速率,在优化出一种较为简单的器件结构后,研究了将其用于组装紫外光电探测器时所需退火温度。实验结果表明,随着后退火工艺温度的增加,紫外光电探测器的EQE从不足5\%(不退火)上升至20\%左右(220℃退火处理)。这一实验结果与\ce{CsPbCl3}全无机钙钛矿薄膜的吸收光谱、XRD图谱以及SEM图片表现出的随退火温度升高薄膜晶粒尺寸增加和结晶性变好的趋势相符合。但将经过较高温度处理后的\ce{CsPbCl3}全无机钙钛矿薄膜应用于紫外光电探测器后,器件的暗电流密度出现骤增。结合对在ZnO上沉积全无机钙钛矿材料的研究,推测出一种可能导致器件性能下降的机制。总之,此项工作为将\ce{CsPbCl3}这种宽带隙全无机钙钛矿材料用于制作高稳定性紫外光电探测器提供了一定研究基础,拓宽了全无机钙钛矿材料在光电器件中的应用范围。

%%%=== 总结与展望 ========%%%

\chapter{总结与展望 }

\section{总结}

本论文以提高基于ZnO/有机半导体复合层的紫外光电探测器的器件性能和制作高稳定性的基于宽带隙全无机钙钛矿的紫外光电探测器为出发点,着眼于对阻碍器件性能提高的有机半导体载流子迁移率较低和全无机钙钛矿前驱体在常规有机溶剂中溶解度较低的问题,从不同的方面解决上述问题,提高了基于基于ZnO/有机半导体复合层的紫外光电探测器的器件性能,并对全无机钙钛矿材料在光电器件中的应用奠定了一定实验基础。

第一章首先介绍了钙钛矿材料的结构、光电特性以及不同形态材料的制备工艺,随后对光电探测的性能参数和常见器件结构进行了简单总结,后续还介绍了钙钛矿材料在光电探测器中的应用,重点介绍了可见光电探测器和紫外光电探测器的研究进展,并分析了紫外光电探测器目前所存在的问题和面临的研究挑战,提出了解决这些问题和挑战的相应方法。

第二章介绍了在PTAA有机半导体中引入\ce{MAPbCl3}钙钛矿量子点对紫外光电探测器性能的影响。\ce{MAPbCl3}钙钛矿量子点通过过饱和溶液重结晶法来制备,并使用含有不同量钙钛矿量子点的甲苯分散液来溶解PTAA粉末的方式对有机半导体进行掺杂,发现在PTAA中适量掺杂钙钛矿量子点不会导致其光吸收发生明显变化。将掺杂不同量钙钛矿量子点的PTAA薄膜与ZnO薄膜复合并用于制作紫外光电探测器,适量量子点的引入对器件暗电流密度没有显著影响,并且器件光电流密度随着钙钛矿量子点掺杂浓度的提高而逐渐增加,但是钙钛矿量子点掺杂浓度太高时器件的暗电流密度会上升大概一个数量级。在进一步优化器件制作工艺后,器件整体的暗电流密度下降大概三个数量级后达到\num{e-9}\ \si{A\ cm^{-2}}(在偏压为-0.5 V下)的水平,并且器件光暗电流密度随钙钛矿量子点掺杂量变化的规律仍与之前相同。向PTAA薄膜中掺杂\ce{MAPbCl3}钙钛矿量子点后,基于此的紫外光电探测器的响应度几乎翻倍,从原来的7.08 \si{mA\ W^{-1}}增加至12.27 \si{mA\ W^{-1}},并且器件的灵敏度也得到一定程度的提高。使用SCLC测试分析,发现掺杂了\ce{MAPbCl3}钙钛矿量子点的PTAA薄膜的平均空穴迁移率$\approx$ \num{1.0e-4}\ \si{cm^2\ V^{-1}\ s^{-1}}(在电场强度为\num{1.0e4} 和\num{1.0e5}\ \si{V \ cm^{-1}} 之间),大约为不含钙钛矿量子点的薄膜载流子迁移率($\approx$ \num{1.5e-5}\ \si{cm^2\ V^{-1}\ s^{-1}})的6倍。总的来看,向PTAA有机半导体中引入适量钙钛矿量子点,不仅会使器件各层之间的能级更加匹配,还提高了有机半导体的载流子迁移率,这都有利于电荷的输运和抽取,进而提高器件性能。此外,还证明了可将此类结构应用于柔性器件。

考虑到有机无机杂化钙钛矿相较全无机钙钛矿存在着稳定性较差的问题,后续还测试了器件的长期稳定性,在放置两个月后,器件的光电流仍为初始值的82\%,展示了较好的长期稳定性。还尝试了将全无机钙钛矿量子点来替换有机无机杂化钙钛矿量子点,但是向PTAA薄膜中掺入\ce{CsPbCl3}钙钛矿量子点,对器件的性能没有明显的影响,其中的原因还需进一步探究。

第三章中研究了\ce{CsPbCl3}全无机钙钛矿材料在紫外光电探测器中的应用。使用热蒸发法制备了\ce{CsPbCl3}全无机钙钛矿薄膜,探索了适用于沉积高质量钙钛矿薄膜的工艺。通过吸收光谱、XRD和SEM来观察不同退火温度下材料的物性变化,发现随着退火温度的增加,\ce{CsPbCl3}钙钛矿薄膜中的晶粒尺寸增加或结晶性变好的趋势,并且没有明显的相变和分解迹象。后续将经过不同后退火温度处理的\ce{CsPbCl3}全无机钙钛矿薄膜应用于紫外光电探测器,研究了器件性能与不同退火温度的关系。实验结果表明,在退火温度$\leq$220℃时,随着退火温度的升高,器件暗电流密度变化幅度较小,光电流密度逐渐增加,但当退火温度为250℃时,器件的暗电流出现了显著上升。通过计算发现,在对热蒸发沉积的全无机钙钛矿薄膜进行适当退火处理后,器件的EQE从不足5\%可上升至20\%左右,这与钙钛矿薄膜形貌和结晶程度随退火温度变化的趋势基本一致。此外,参考在不同ZnO薄膜上沉积钙钛矿薄膜的研究,推测出一种经过较高退火温度处理后可导致器件性能变差的原因。\ce{CsPbCl3}薄膜和ZnO薄膜之间可通过Cs-Ac之间相互作用来抑制钙钛矿薄膜中的缺陷,随着退火温度的增加,ZnO薄膜中的\ce{Ac^-}逐渐挥发减少,对钙钛矿中缺陷抑制作用减弱,从而导致\ce{CsPbCl3}薄膜在较高温度退火后再应用于紫外光电探测器会获得较差的器件性能。

\section{展望}

在将宽带隙钙钛矿材料应用到紫外光电探测器研究探索中,虽然取得了一些研究成果,提升了器件性能和拓宽了全无机钙钛矿材料在光电器件中的应用范围,但仍存在一些问题需要后续研究来解决:

(1)有机半导体在掺杂有机无机杂化钙钛矿材料后器件性能得以提高,而使用全无机钙钛矿掺杂却对器件性能影响不大,这背后的物理机制需要进一步猜想和验证;

(2)虽然可使用热蒸发的工艺来解决全无机钙钛矿由于溶解度较低较难使用溶液旋涂法制备的问题,但实际使用热蒸发设备时发现,即使在制作条件相近的条件下,所得器件的性能差异也会较大,为得到切实可靠的实验规律需要进行大量的重复实验,这些问题一定程度上是人工操作设备引起的。因此迫切需要人为干扰更小的自动化设备来投入材料制备和器件制作,这不仅有利于科学研究,也会推动钙钛矿材料的商业化应用;

(3)在使用宽带隙钙钛矿材料制作紫外光电器件时,所用电子传输层和空穴传输层大都是从在钙钛矿太阳能电池或钙钛矿可见光电探测器中广泛应用的材料中挑选的,虽然器件可以工作,但是为进一步提高器件性能,或需要精心挑选更合适的电子传输层和空穴传输层材料。

随着对钙钛矿材料研究的不断发展,钙钛矿材料在光电器件中的应用越来越成熟,很多性能指标已达到商业化应用标准。有理由相信钙钛矿材料作为最具有应用前景的半导体材料会在我们日常生活中得到广泛应用。}


%%%=== 参考文献 ========%%%
\cleardoublepage\phantomsection
\addcontentsline{toc}{chapter}{参考文献}
%\printbibliography[heading=bibliography,title=参考文献]
\bibliography{reference-5.13}%参考文献中存在下标可用\textsubscript{}改写,参考文献中存在上标可用\textsuperscript{}改写。2020/1/21 Lihao Cui

%参考文献输出的文章标题默认是首字母大写,对需要大写的字母可用{}括起来
%对于文章标题中化学式的书写可参考PRB这类期刊中输出的格式,常用到\mathrm{}_{}
%如无特殊要求,需要在EndNote中将文献的DOI和URL删除
%科研成果可复制Word输出的参考文献文本,并对需要加粗或添加下划线的部分进行重新编辑


\backmatter
% !Mode:: "TeX:UTF-8"

%%% 此部分内容:  (1) 致谢  (2) 武汉大学学位论文使用授权协议书(无需改动)

%%%%%%%%%%%%%%%%%%%%%%%
%%%------- 攻博(硕)期间科研成果 -------%%%
%%%%%%%%%%%%%%%%%%%%%%%
\reseachresult

\begin{enumerate}[{[1]}]
\item  \textbf{\underline{Lihao Cui}}, Jiali Peng, Wei Li, Yalun Xu, Meijuan Zheng, Qianqian Lin. Ultraviolet photodetectors based on \ce{CH3NH3PbCl3} perovskite quantum dots-doped poly(triarylamine)[J]. \textbf{Physica Status Solidi-Rapid Research Letters}, 2020, 14(4): 1900653.

\item Jiali Peng, \textbf{\underline{Lihao Cui}}, Ruiming Li, Yalun Xu, Li Jiang, Yuwei Li, Wei Li, Xiaoyu Tian, Qianqian Lin. Thick junction photodiodes based on crushed perovskite crystal/polymer composite films[J]. \textbf{Journal of Materials Chemistry C}, 2019, 7(7): 1859-1863.

\item Jiali Peng, Kuangkuang Ye, Yalun Xu, \textbf{\underline{Lihao Cui}}, Ruiming Li, Hao Peng, Qianqian Lin. X-ray detection based on crushed perovskite crystal/polymer composites[J]. \textbf{Sensors and Actuators A: Physical}, 2020, 312: 112132.

\item	Li Jiang, Yuwei Li, Jiali Peng, \textbf{\underline{Lihao Cui}}, Ruiming Li, Yalun Xu, Wei Li, Yanyan Li, Xiaoyu Tian, Qianqian Lin. Solution-processed \ce{AgBiS2} photodetectors from molecular precursors[J]. \textbf{Journal of Materials Chemistry C}, 2020, 8(7): 2436-2441.

\item Jiali Peng, Chelsea Q. Xia, Yalun Xu, Ruiming Li, \textbf{\underline{Lihao Cui}}, Jack K. Clegg, Laura M. Herz, Michael B. Johnston, Qianqian Lin. Crystallization of \ce{CsPbBr3} single crystals in water for X-ray detection[J]. \textbf{Nature Communications}, 2021, 12(1): 1531.

\item	Ruiming Li, Jiali Peng, Yalun Xu, Wei Li, \textbf{\underline{Lihao Cui}}, Yanyan Li, Qianqian Lin. Pseudohalide additives enhanced perovskite photodetectors[J]. \textbf{Advanced Optical Materials}, 2021, 9(2): 2001587.

\item	Wei Li, Yalun Xu, Jiali Peng, Ruiming Li, Jiannan Song, Huihuang Huang, \textbf{\underline{Lihao Cui}}, Qianqian Lin. Evaporated perovskite thick junctions for X-ray detection[J]. \textbf{ACS Applied Materials \& Interfaces}, 2021, 13(2): 2971-2978.

\item	Wei Li, Yalun Xu, Xianyi Meng, Zuo Xiao, Ruiming Li, Li Jiang, \textbf{\underline{Lihao Cui}}, Meijuan Zheng, Chang Liu, Liming Ding, Qianqian Lin. Visible to near-infrared photodetection based on ternary organic heterojunctions[J]. \textbf{Advanced Functional Materials}, 2019, 29(20): 1808948.

\item 林乾乾, {\bf\songti{\underline{崔立豪}}},一种钙钛矿量子点掺杂的有机紫外探测器及其制备方法:CN111009613A[P],2020.04.14。

\end{enumerate}
%%%%%%%%%%%%%%%%%%%%%%%
%%% --------------- 致谢 ------------- - %%%
%%%%%%%%%%%%%%%%%%%%%%%
\acknowledgement
%感谢你、感谢他、感谢大家。
%本论文是在我的导师林乾乾教授悉心指导和帮助下完成的,在此我由衷地表示感谢!林老师作为我科研道路上的领路人和启蒙者,专业知识雄厚、治学态度严谨,让我在科学研究探索道路上受益匪浅。同时,我有幸成为林老师第一届研究生,在我读硕期间跟随林老师搭建实验室,在林老师手把手指导下做实验、写论文、写专利等,让我学到了很多知识和技能。
{\songti\zihao{-4}三年研究生时光匆匆而过,不知是研究生阶段的忙碌生活还是突如其来的新冠疫情加快了时间的流逝。细想来我与武大结缘已七年有余,似乎冥冥之中注定我就要在这里走一遭。

本论文的完成,离不开我的导师林乾乾教授的悉心指导和帮助,在此我由衷地表示感谢!林老师思路开阔、作风勤勉、态度严谨,为我们创造了良好的实验条件和研究氛围。读硕期间跟随林老师搭建实验室,在林老师手把手指导下做实验、写论文、写专利等等,让我不仅丰富了知识还提高了动手能力。在李哲师兄的介绍下,我有幸成为林老师的第一届研究生,在此也表示对李哲师兄的感谢!
%感谢桂鹏彬师兄、姚方师兄,是他们帮助我尽快熟悉实验室环境,引导我开展实验。
%三年研究生时光匆匆而过,不知是研究生阶段的忙碌生活还是突如其来的新冠疫情加快了时间的流逝。细想来我与武大结缘已七年有余,当年我在高中电脑机房选择了武汉大学物理科学与技术学院,就注定我要在这里走一遭。

%在大四上学期时我就找到了林乾乾老师,林老师是我们学院新进教师,我也有幸成为林老师的第一届研究生,随后跟随林老师搭建实验室,在林老师手把手指导下做实验、写论文、写专利等等,让我学习到了很多知识和技能。研究生阶段的顺利度过和本论文的完成,我首先要感谢林老师指导与帮助。我还要感谢我的父母,虽然父母文化水平不高、家里生活条件也不是很好,但是父母还是一如既往地支持我读书求学。我还要感谢李睿明、李威和许亚伦,从东湖新村出租屋

本论文使用~\LaTeX~书写,要感谢武汉大学黄正华老师提供的~\LaTeX~使用教程和硕士研究生毕业论文模板,感谢本科室友李云飞同学提供的学习资源,还要感谢清华大学李泽平在参考文献格式方面提供的指导,他们的帮助使我极大地提高了论文书写效率。

感谢方国家老师课题组在实验条件上所提供的支持,并感谢方老师课题组的桂鹏彬师兄和姚方师兄,是他们帮助我尽快熟悉实验室环境,引导我顺利开展实验。还要感谢我们学院的刘雍老师、蒲十周老师在实验测试分析方面提供的指导。

三年研究生时光短暂而快乐,离不开同课题组兄弟姐妹的陪伴与支持。感谢和我同级的李睿明、李威和许亚伦,他们从不吝啬分享生活中的趣事和科研上的心得;感谢彭家丽师姐在测试分析上提供的热心帮助;此外,还有江力师兄,李雨薇、李妍妍、郑美娟、田晓语等师姐,黄辉煌、齐一鸣、余甜、桂福兵、宋建楠、喻岚鑫、白颂雪等师弟师妹们,感谢他们一同创造的融洽氛围,让我度过三年快乐的时光。

%感谢我的朋友,长久以来一直激励我勇敢前行 

感谢我的父母和亲人,还有小璐同志,感谢他们一如既往的支持和关心,为我在求学路上提供最坚实的依靠。

感谢各位专家、老师,感谢他们在百忙之中参与审阅和答辩。}

%%%%%%%%%%%%%%%%%%%%%%%%%%%%%%%%%%%%%%%
%%%%%%%--判断是否需要空白页-----------------------------
  \iflib
  \else
  \newpage
  \cleardoublepage
  \fi
%%%%%%%-------------------------------------------------







 %%%硕士期间发表科研成果,致谢。
\cleardoublepage
\end{document}



